\chapter{Уравнение Нернста--Планка}
\section{Основные положения}

Если считать мембрану гомогенной средой, в которой может происходить диффузия и
миграция ионов, парциальный электрический ток ионов сорта к можно записать в
виде \cite{bib:8}:
\begin{equation}
    I_k = -\frac{z_k}{\abs{z_k}}u_kRT\pder{c_k}{x} + \abs{z_k}u_kc_kFE,
    \label{eq:nernst-plank}
\end{equation}
где \( z_k \) -- заряд иона сорта к (в единицах заряда протона), \( u_k \) --
подвижность, Е -- напряженность электрического поля.
Суммарная плотность ионного тока определяется как сумма
\[
    I = \sum_{k=1}^n I_k.
\]
В стационарном случае парциальные токи сохраняются, так что \( I_k = \const \),
и соотношение \eqref{eq:nernst-plank} представляет собой нелинейное
дифференциальное уравнение первого порядка, содержащее неизвестные функции
\( c_k \) и \( E \) и неизвестную постоянную \( I_k \). Это основное
электродиффузионное уравнение носит название уравнения Нернста -- Планка
\cite{bib:16}.
Если рассматриваемая система содержит \( n \) сортов ионов, то мы имеем \( n \)
уравнений \eqref{eq:nernst-plank} для \( n + 1 \) функций, в число которых
входят все \( c_k \) и \( Е \). Чтобы сделать задачу определенной, необходимо
располагать еще одним уравнением. Таким уравнением служит уравнение Пуассона.
Ввиду важности выпишем систему уравнений электродиффузионной задачи:
\begin{equation}
\left\{
    \begin{array}{l}
        \ds \der{c_k}{x} - \beta z_k c_k e E =
            - \frac{I_k\abs{z_k}}{z_k u_k RT},\\
        \ds \der{E}{x} = \frac{F}{\eps\eps_0}\sum_{k=1}^n z_k c_k.
    \end{array}
\right.
\label{eq:system_nernst-plank}
\end{equation}
Задавая \( 2n + 2 \) граничных условий, мы получаем окончательную постановку
проблемы.

Кроме дифференциальной формулировки электродиффузионных уравнений
\eqref{eq:nernst-plank} и \eqref{eq:system_nernst-plank} иногда оказывается
полезной интегральная. Проинтегрируем уравнение Пуассона от некоторой точки
\( \overline{x} \) до \( x \) \cite{bib:8}:
\begin{equation}
    E(x, t) = E(\overline{x}, t) + \frac{F}{\eps}\sum_{k=1}^n z_k
    \int_{\overline{x}}^x c_k(x', t) dx'.
    \label{eq:electric_field}
\end{equation}
Подставляя \eqref{eq:electric_field} в выражение для разности потенциалов на мембране
\[
    \phi(t) = -\int_0^\delta E(x', t) dx',
\]
получаем
\[
    \phi(t) = -\delta E(\overline{x}, t) - \frac{F}{\eps}\sum_{k=1}^n z_k
    \int_0^\delta \int_{\overline{x}}^{x'} c_k(x'', t) dx'' dx'.
\]
Подставим \(Е(\overline{x}, t)\) в \eqref{eq:electric_field}:
\begin{equation}
    E(x, t) = -\frac{\phi(t)}{\delta} -
    \frac{F}{\eps\delta}\sum_{k=1}^n z_k
    \int_0^\delta \int_{\overline{x}}^{x'} c_k(x'', t) dx'' dx' +
    \frac{F}{\eps}\sum_{k=1}^n z_k\int_{\overline{x}}^x c_k(x', t) dx'.
\end{equation}
Найдем теперь ток смещения
\[
    \eps\pder{E}{t} = -\frac{\eps}{\delta}\der{\phi}{t} -
    \frac{F}{\delta}\sum_{k=1}^n z_k
    \int_0^\delta \int_{\overline{x}}^{x'} \pder{c_k(x'', t)}{t} dx'' dx' +
    F\sum_{k=1}^n z_k\int_{\overline{x}}^x \pder{c_k(x', t)}{t} dx'.
\]
Из условия непрерывности:
\[
    F\sum_{k=1}^n z_k\pder{c_k}{t} = -\pder{I}{x}.
\]
Следовательно,
\begin{gather*}
    -\frac{\eps}{\delta}\der{\phi}{t} +
    \frac{1}{\delta}\int_0^\delta \int_{\overline{x}}^{x'} \pder{I}{x''} dx'' dx' -
    \int_{\overline{x}}^x \pder{I}{x'} dx' = \\ =
    -\frac{\eps}{\delta}\der{\phi}{t} +
    \frac{1}{\delta}\int_0^\delta [I(x', t) - I(\overline{x}, t)] dx' -
    [I(x, t) - I(\overline{x}, t)] = \\ = -\frac{\eps}{\delta}\der{\phi}{t} +
    \frac{1}{\delta}\int_0^\delta I(x', t) dx' - I(x, t).
\end{gather*}
Суммарная плотность тока равна
\[
    I_0(t) = -\frac{\eps}{\delta}\der{\phi}{t} +
        \frac{1}{\delta}\int_0^\delta I(x', t) dx'.
\]

Другое интересное соотношение получается, если разделить \eqref{eq:nernst-plank}
почленно на \( c_k(x) \) и проинтегрировать по всей толщине мембраны:
\begin{gather*}
    \int_0^\delta \frac{l_k(x', t)}{c_k(x', t)} dx' =
    \beta z_k \int_0^\delta E(x', t) dx' - \int_{c_k(0)}^{c_k(\delta)} d\ln c_k
    =\\= -\beta z_k \phi(t) + \ln\frac{c_k(0)}{c_k(\delta)} =
    -\beta z_k[\phi(t) - \phi_k],
\end{gather*}
где \( l_k = I_k / z_k u_k RT \), а \( \phi_k \)-- равновесный мембранный
потенциал для ионов сорта \( k \), определяемый соотношением Нернста
\[
    \phi_k = \frac{1}{z_k\beta}\ln\frac{c_k(0)}{c_k(\delta)}.
\]
В стационарном состоянии \( \phi(t) = \phi \) и \( l_k = \const \), так что
\begin{equation}
    I_k = z_k u_k RT l_k = -g_k(\phi - \phi_k),
    \label{eq:CVC}
\end{equation}
где парциальная проводимость \( g_k \) для ионов сорта \( k \) определена
формулой
\begin{equation}
    g_k = \frac{F z_k^2 u_k}{\int_0^\delta\frac{dx'}{c_k(x')}}.
    \label{eq:conductivity}
\end{equation}

Вообще говоря, \( g_k \) зависит от потенциала через концентрационный профиль
под знаком интеграла в \eqref{eq:conductivity}, так что вольтамперная
характеристика \eqref{eq:CVC} является нелинейной. Тем не менее запись
проводимости в форме \eqref{eq:CVC}, где выделен линейный по смещению потенциала
сомножитель, имеет определенный смысл. Во-первых, при вычислении проводимости в
пределе малого поля, как следует из \eqref{eq:conductivity}, можно использовать
равновесный концентрационный профиль, после чего решение сводится к одной
квадратуре. Во-вторых, при быстром изменении внешнего поля распределение
концентраций в мембране не успевает перестроиться, а поэтому на малых временах
формула \eqref{eq:CVC} выражает линейную зависимость парциального ионного тока
от сдвига потенциала относительно равновесного значения. Следует обратить
внимание на определенную аналогию между соотношением \eqref{eq:CVC} и
эмпирическими уравнениями Ходжкина -- Хаксли для ионных токов через биомембраны,
которые также имеют омический характер на малых временах.

\section{Приближенное решение Планка}
В точной формулировке \eqref{eq:system_nernst-plank} уравнения ионного
транспорта приводят к трудно обозримым результатам; поэтому в большинстве
случаев пользуются приближенными решениями, основанными на тех или иных
предположениях.

Начнем с краткого изложения приближения Планка, которое состоит в том, что
в мембране, по предположению, выполняется условие электронейтральности. Таким
образом, основная система уравнений электродиффузионной задачи
\eqref{eq:system_nernst-plank} сводится к приближенной системе
\begin{equation}
    \left\{
    \begin{array}{l}
        \ds \der{c_k}{x} - \beta z_k c_k e E =
            - \frac{I_k\abs{z_k}}{z_k u_k RT},\\
        \ds \der{E}{x} = \frac{F}{\eps\eps_0}\sum_{k=1}^n z_k c_k = 0.
\end{array}
\right.
\label{eq:system_nernst}
\end{equation}
в которую вместо уравнения Пуассона входит условие электронейтральности. Начало
координат совместим с левой границей мембраны. Граничные условия для
концентрации электролита в мембране зададим в виде
\begin{equation}
    \left.c_k\right|_{x=0} = c_k(0), \left.c_k\right|_{x=\delta} = c_k(\delta).
    \label{eq:boundary_conditions}
\end{equation}
Задача станет полностью определенной, если известны либо разность потенциалов,
приложенная к мембране \( \phi(0) - \phi(\delta) \), либо электрический ток.
В общем случае смеси электролитов сложного состава решение
\eqref{eq:system_nernst} связано со значительными трудностями. Поэтому, чтобы
выявить основные качественные особенности планковского приближения, рассмотрим
простейший пример бинарного электролита \( A^+B^- \), концентрации которого
слева и справа от мембраны различны. Воспользовавшись условием
электронейтральности, а затем складывая и вычитая почленно уравнения
Нернста -- Планка для анионов и катионов, сведем \eqref{eq:system_nernst} к
следующей системе
\begin{align}
    & \der{c}{x} = \chi,              \label{eq:system_plank_binary_1}\\
    & c(x)\der{\phi}{x} = -\alpha,    \label{eq:system_plank_binary_2}
\end{align}
где
\begin{equation}
    \chi = \frac{I_B u_A - I_A u_B}{2Ru_Au_B},\quad
    \alpha = \frac{I_B u_A + I_A u_B}{2Ru_Au_B\beta}.
    \label{eq:system_plank_binary_subs}
\end{equation}
Отсюда следует, что распределение концентрации электролита в мембране в
планковском приближении характеризуется постоянным наклоном. Интегрируя
\eqref{eq:system_plank_binary_1} с граничным условием
\eqref{eq:boundary_conditions}, получаем линейный концентрационный профиль
\begin{equation}
    c(x) = \chi x + c(0).
    \label{eq:plank_binary_conc}
\end{equation}
Полагая \( x = \delta \), находим связь между параметром \( \chi \) и градиентом
концентрации:
\begin{equation}
    \chi = \frac{c(\delta)-c(0)}{\delta}.
    \label{eq:plank_binary_chi}
\end{equation}
Проинтегрируем теперь уравнение \eqref{eq:system_plank_binary_2},
воспользовавшись \eqref{eq:plank_binary_conc}:
\begin{equation}
    \phi(x) = -\frac{\alpha}{\chi}\ln[\chi х + c(0)] + \const.
    \label{eq:plank_binary_pot}
\end{equation}
С помощью \eqref{eq:plank_binary_pot} находим разность потенциалов \( \phi \),
определенную как \( \phi(0) - \phi(\delta) \):
\begin{equation}
    \phi = \frac{\alpha}{\chi}\ln\frac{\chi\delta + c(0)}{c(0)}.
    \label{eq:plank_binary_phi}
\end{equation}
Выразим здесь \( \chi \) с помощью \eqref{eq:plank_binary_chi} и перепишем
\eqref{eq:plank_binary_phi} в виде:
\begin{equation}
    \alpha = \frac{\phi[c(\delta) - c(0)]}{\delta\ln[c(\delta) / c(0)]}.
    \label{eq:plank_binary_alpha}
\end{equation}
Из \eqref{eq:system_plank_binary_subs} выразим \(I_A\) и \(I_B\) через
параметры \( \alpha \) и \( \chi \):
\begin{equation}
    \begin{array}{l}
        I_A = RТu_A (\alpha\beta - \chi),\\
        I_B = RTu_B (\chi - \alpha\beta),
    \end{array}
    \label{eq:plank_binary_currents}
\end{equation}
а затем, подставив в \eqref{eq:plank_binary_currents} выражения
\eqref{eq:plank_binary_chi} и \eqref{eq:plank_binary_alpha}, найдем ионные
токи
\begin{equation}
    I_k = RTu_k\frac{c(\delta) - c(0)}{\delta\ln[c(\delta)/c(0)]}\cdot
    \left[\psi - z_k\ln\frac{c(\delta)}{c(0)}\right],
    \label{eq:plank_binary_currents_2}
\end{equation}
\[
    k = A, B,\quad z_A = 1,\quad z_B = -1.
\]
Полный электрический ток равен:
\begin{equation}
    I = I_A + I_B = \frac{RT[c(\delta) - c(0)]}{\delta\ln[c(\delta)/c(0)]}\cdot
    \left[(u_A + u_B)\psi + (u_B - u_A)\ln\frac{c(\delta)}{c(0)}\right].
    \label{eq:plank_binary_current}
\end{equation}
Приравнивая электрический ток нулю, найдем диффузионный потенциал, который
устанавливается на мембране в условиях разомкнутой цепи:
\begin{equation}
    \psi_d = \frac{u_A - u_B}{u_A + u_B}\ln\frac{c(\delta)}{c(0)}.
    \label{eq:plank_binary_psid}
\end{equation}
Последняя формула физически вполне прозрачна. Если анионы и катионы обладают
разными подвижностями, то на мембране возникает электрическое поле,
компенсирующее эту разницу. Разность потенциалов \eqref{eq:plank_binary_psid},
конечно, не является термодинамически равновесной. Благодаря различию
концентраций слева и справа от мембраны через нее идет постоянный поток
электролита. Он отличен от нуля и при разомкнутой цепи, когда
\( \psi = \psi_d \). Но только в этом случае парциальные токи анионов и катионов
равны друг другу. Таким образом, диффузионный потенциал возникает как следствие
неравновесности системы, которая выражается в том, что \( c(\delta) \neq c(0) \)
при условии, что \( u_A \neq u_B \). Если подвижности анионов и катионов
совпадают, то присущая системе неравновесность электрически не проявляется --
диффузионный потенциал не возникает.

Как следует из \eqref{eq:plank_binary_current}, электрический ток
пропорционален отклонению приложенной разности потенциалов от диффузионного
потенциала. Чтобы сделать это более наглядным, перепишем
\eqref{eq:plank_binary_currents_2} в следующем виде:
\begin{equation}
    I_k = RTu_k\frac{c(\delta) - c(0)}{\delta\ln[c(\delta)/c(0)]}(\psi-\psi_k),
    \quad k = A,B.
\end{equation}

Каждый из ионных токов обращается в нуль только при таком значении внешнего
потенциала, которое совпадает с его равновесным потенциалом. Но ионы разных
зарядов имеют различные равновесные потенциалы. Так, нернстовские потенциалы
анионов и катионов прямо противоположны по знаку. Поэтому в системе может
установиться лишь стационарное состояние, зависящее от конкретных внешних
условий, в котором не реализуются парциальные ионные равновесия.

Согласно \eqref{eq:plank_binary_pot}, потенциал в мембране изменяется по
логарифмическому закону. Это распределение характеризуется отличной от нуля
второй производной по координате, которая связана с плотностью заряда уравнением
Пуассона. Таким образом, исходя из условия электронейтральности, мы нашли
приближенное решение, которое приводит к выводу о наличии в системе отличного от
нуля объемного заряда. Малость этого заряда по сравнению с концентрацией ионов
служит критерием применимости условия электронейтральности. Планковское описание
ионного транспорта с успехом применяется в случае мембран, толщина которых много
больше размеров диффузных обкладок двойных слоев, находящихся в мембране у
границ раздела.

Планковское решение легко обобщается на случай бинарного электролита с ионами
произвольных зарядов, когда условие электронейтральности имеет вид
\( z_Ac_A + z_Bc_B = 0\). Суммарная плотность тока равна
\begin{equation}
    I = \frac{z_ART[c(\delta) - c(0)]}{\delta\ln[c(\delta)/c(0)]}\cdot
    \left[(z_Au_A - z_Bu_B)\psi + (u_B - u_A)\ln\frac{c(\delta)}{c(0)}\right].
\end{equation}
а диффузионный потенциал
\begin{equation}
    \psi_d = \frac{u_A - u_B}{z_Au_A - z_Bu_B}\ln\frac{c(\delta)}{c(0)}.
\end{equation}

В экспериментах часто встречается случай смеси двух электролитов с общим ионом,
например АВ и ХВ. При произвольных \( z_A, z_X, z_B \) задача является
нелинейной. Она линеаризуется в том случае, если какие-либо две зарядности
совпадают, например \( z_A = z_X = z \). Выражение для тока в этом случае имеет
вид:
\begin{gather}
    I = RT\frac{c_B(0) - c_B(\delta)}{\delta}\left\{
        z_Bu_B\left(1-\frac{z_B\psi}{\ln[c_B(\delta)/c_B(0)]}\right)
        + \right.\\ +\left.
        z\left[e^{-z\psi}[u_Ac_A(\delta) + u_Xc_X(\delta)] -
        [u_Ac_A(0) + u_Xc_X(0)]\right]
        \left(1-\frac{z\psi}{\ln[c_B(\delta)/c_B(0)]}\right)\right\}
        \label{eq:plank_ternary_current}
\end{gather}

Диффузионный потенциал \( \psi_d \) получается из
\eqref{eq:plank_ternary_current} при условии \( I = 0 \) как решение
трансцендентного уравнения. Пусть p -- число различных зарядностей,
\( \tilde{c}_k \) -- концентрация всех ионов зарядности \( z_k \),
\( c = \sum c_k \) -- суммарная концентрация. Тогда для парциального тока имеем
\begin{equation}
    I_k = z_ku_kRT\frac{c(0) - c(\delta)}{\delta}\cdot
    \frac{c_k(\delta)e^{-z_k\psi}-c_k(0)}
        {\tilde{c}_k(\delta)e^{-z_k\psi}-\tilde{c}_k(0)}\cdot
    \frac{\sum_{i=1}^{p-1}[1/z_k - f_i(\psi)]}
        {\prod_{i=1,\ i \neq k}(1/z_k - 1/z_i)},
\end{equation}
где функции \( f(\psi) \) удовлетворяют трансцендентному уравнению
\begin{equation}
    \exp(-\psi) = \left[\sum_{k=1}^p\frac{z_k\tilde{c}_k(0)}{f-1/z_k}\right]^f
    \cdot \left[\sum_{k=1}^p\frac{z_k\tilde{c}_k(\delta)}{f-1/z_k}\right]^{-f},
\end{equation}
имеющему \( p-1 \) решений. Отыскав ионные токи, можно найти суммарный
электрический ток и диффузионный потенциал.

\section{Гольдмановское приближение постоянного поля}
В случае тонких мембран, когда длина экранирования превосходит толщину мембраны,
предположение электронейтральности теряет силу. Более убедительным является
здесь приближение постоянного поля \cite{bib:17}. Точная система уравнений
электродиффузионной задачи \eqref{eq:system_nernst-plank} в гольдмановском
приближении имеет вид:
\[
    \der{c_k}{x} - \beta z_k c_k e E = - \frac{I_k\abs{z_k}}{z_k u_k RT},
    \quad E = \const.
\]
Граничными условиями, как и в предыдущем разделов, являются заданные на краях
мембраны значения концентрации ионов \( c_k(0) \) и \( c_k(\delta) \). Условие
постоянства поля приводит к линеаризации уравнений Нернста -- Планка, которые
теперь легко интегрируются. Для ионного тока получается выражение
\begin{equation}
    I_k = \frac{z_k^2 RT u_k \psi}{\delta}\cdot
        \frac{c_k(0) - c_k(\delta)e^{-z_k\psi}}{1 - e^{-z_k\psi}},
        \label{eq:goldman_currents}
\end{equation}
а концентрационный профиль имеет вид
\[
    c_k(x) = c_k(0) +
    [c_k(\delta) - c_k(0)]\frac{e^{z_k\beta Ex} - 1}{e^{z_k\beta E\delta} - 1}.
\]
Здесь разность потенциалов на мембране \( \phi \) по-прежнему определена как
\( \phi(0) - \phi(\delta) \), а \( E = \phi/\delta \). В отличие от планковского
случая профиль концентраций теперь нелинеен по х и зависит от поля. Если поле
положительно, то на катионы (кроме концентрационного градиента) действует
электрическая сила, направленная в ту же сторону, что и поле. Поэтому при
\( c_k(0) > c_k(\delta) \) профиль концентрации оказывается выпуклым. При
обратном знаке поля профиль будет вогнутым. С увеличением абсолютной величины
приложенного поля концентрация почти в о всех точках мембраны становится такой
же, как на левой шли правой границе в зависимости от знака поля.

Зависимость ионных токов \eqref{eq:goldman_currents} от приложенной разности
потенциалов в гольдмановском случае нелинейна. Только в симметричных условиях,
когда концентрации ионов слева и справа от мембраны одинаковы, функция
\eqref{eq:goldman_currents} становится линейной. Парциальные ионные токи
обращаются в нуль при таком значении внешнего потенциала, которое совпадает с
парциальным нернстовским потенциалом.

В условиях разомкнутой цепи в системе возникает мембранный потенциал, который
можно найти, приравняв нулю суммарный электрический ток. Выпишем результат для
случая бинарного 1 : 1 электролита:
\begin{equation}
    \psi_G =
        \ln\frac{u_A c_A(\delta) + u_B c_B(0)}{u_A c_A(0) + u_B c_B(\delta)}.
    \label{eq:goldman_psi}
\end{equation}
Физическое содержание этой формулы такое же, как и \eqref{eq:plank_binary_psid}
для планковского потенциала. Гольдмановский потенциал возникает как следствие
неравновесности системы. По своей структуре формула \eqref{eq:goldman_psi},
конечно, отличается от \eqref{eq:plank_binary_psid}, так как в основу ее вывода
было положено условие постоянства поля, а не условие электронейтральности в
каждой точке мембраны.

Если мембрана проницаема для ионов одного знака, причем заряды всех ионных
компонентов одинаковы, то формулу типа \eqref{eq:goldman_psi} для мембранного
потенциала можно получить, не требуя постоянства поля. Суммируя уравнения
Нернста--Планка \eqref{eq:nernst-plank} по всем \( k \), с учетом условия
\( \sum I_k= 0 \) получим
\begin{equation}
    d\psi = -\frac{\sum z_k u_k dc_k}{\sum z_k^2 u_k c_k}.
    \label{eq:difpot}
\end{equation}
Поскольку заряды всех ионов одинаковы, то из \eqref{eq:difpot} получаем
\[
    d\psi = -\frac{d\sum u_k c_k}{z \sum u_k c_k},
\]
или
\[
    d\psi = -\frac{1}{z}d\ln\left(\sum u_k c_k\right),
\]
откуда непосредственно следует
\begin{equation}
    \psi_G =\frac{1}{z}\ln\frac{\sum u_k c_k(\delta)}{\sum u_k c_k(0)}.
    \label{eq:difpot1}
\end{equation}
Гольдмановское приближение, в частности выражение для мембранного потенциала,
широко используется при описании ионного транспорта через биологические
мембраны. Однако формулы \eqref{eq:goldman_psi} и \eqref{eq:difpot1} непосредственно не
могут быть применены для обработки экспериментального материала, так как в них
входят неизвестные концентрации в мембране у ее границ, тогда как задаваемой
величиной является концентрация ионов в объеме окружающих ее растворов. Кроме
того, не следует забывать, что потенциал \( \psi_G \) представляет только
внутреннюю, собственно мембранную часть полной мембранной разности потенциалов,
в которую входят еще равновесные поверхностные скачки. Считая, что мембрана не
имеет фиксированного поверхностного заряда и ее толщина меньше длины
экранирования, можно пренебречь поверхностными скачками потенциала, которые в
этих предположениях малы, и выразить концентрации в мембране через объемные
значения с помощью коэффициентов распределения. Тогда \eqref{eq:goldman_psi}
приобретает вид
\begin{equation}
    \psi_G =
    \ln\frac{P_A c_{A2} + P_B c_{B1}}{P_A c_{A1} + P_B c_{B2}},
    \label{eq:goldman_psi_P}
\end{equation}
где введены проницаемости
\( P_k = u_k\gamma_k/\beta\delta = \gamma_k D_k / \delta \). Формула
\eqref{eq:goldman_psi_P}, известная как соотношение Гольдмана--Ходжкина--Катца,
определяет полный мембранный потенциал через концентрации ионов в растворах
\( c_{ij} \).

Если поверхностными скачками потенциала нельзя пренебречь, то полный мембранный
потенциал можно представить в виде суммы
\[
    \psi = \psi_1 + \psi_2 + \psi_3.
\]
Величина \( \psi_2 \) определяется формулой \eqref{eq:goldman_psi_P}, а \( \psi_1 \) и
\( \psi_3 \) можно связать с концентрациями и коэффициентами распределения:
\[
    \psi_1 = \frac{1}{z}\ln\frac{c_k(0)}{\gamma_k c_{k1}},\quad
    \psi_3 = \frac{1}{z}\ln\frac{c_k(\delta)}{\gamma_k c_{k2}}.
\]
Отсюда,
\[
    \psi =\frac{1}{z}
        \ln\frac{\sum u_k \gamma_k c_{k2}}{\sum u_k \gamma_k c_{k1}}.
\]
При выводе этой формулы, которая связывает полный трансмембранный скачок
потенциала с объемными концентрациями электролита, предполагалось, что мембрана
проницаема только для ионов одного знака. Оправдать «запирание» мембраны для
ионов противоположного знака можно в том случае, если она несет значительный
фиксированный заряд, вытесняющий из мембраны ко-ионы.

Приближение постоянного поля предполагает наличие равновесного распределения ионов
на границе мембраны с раствором, т. е. лимитирующая стадия ионного транспорта тем
самым автоматически перемещается в объем мембраны. Однако это допущение отнюдь
не является критическим и его легко избежать. Предположим, что мембрана
проницаема для ионов только одного типа, которые входят в мембрану из раствора
с константой скорости \(k_1\) и выходят из нее с константой \(k_2\). Величины
\(k_1\) и \(k_2\), вообще говоря, могут зависеть от приложенного потенциала.
Решение уравнений ионного транспорта в приближении постоянного поля
дает следующую вольтамперную характеристику:
\begin{equation}
    I = zFk_1\frac{c_1e^{z\psi}-c_2}
        {e^{z\psi}+1+\frac{\delta k_2}{z\psi D_m}(e^{z\psi}-1)}.
    \label{eq:120}
\end{equation}
Предельные токи равны соответственно
\[
    I_1 = zFk_1c_1, \quad I_2 = -zFk_1c_2.
\]
В эти выражения входят только характеристики границы. Увеличивая разность
потенциалов на мембране, мы ускоряем объемную стадию переноса ионов и в конце
концов она становится настолько быстрой, что перестает лимитировать мембранный
транспорт. В результате член, пропорциональный \( D_m / \delta \), исчезает из
формулы для вольтамперной характеристики. Любопытно сравнить выписанные выше
предельные токи с аналогичными выражениями, полученными для
неперемешиваемых слоев. По структуре формулы очень похожи, а при замене
\( D/h \to k_1 \) они просто переходят друг в друга. Таким образом, роль границ
мембраны с раствором может быть вполне аналогичной роли неперемешиваемых слоев.
Это обстоятельство будет использовано в модели переносчиков.

Отличие вольтамперной характеристики \eqref{eq:120} от гольдмановской состоит еще и в
том, что здесь уже нельзя ввести понятие проницаемости Р как множителя,
содержащего константы скоростей переноса и определяющего амплитуду вольтамперной
характеристики. Величины \( D_m/\delta \), \( k_1 \) и \( k_2 \) входят в более
сложных комбинациях с функциями потенциала. Формула \eqref{eq:120} является более
общей. Гольдмановская вольтамперная характеристика получается из нее как частный
случай, если скорости граничных переходов \( k_1 \) и \( k_2 \) устремить к
бесконечности, оставляя их отношение, равное коэффициенту распределения
\( k_1/k_2= \gamma \), конечным.

В симметричных системах вольтамперная характеристика \eqref{eq:120} также становится
симметричной:
\[
    I = \frac{z^2FD_mk_1c\psi}{\delta}\cdot
    \frac{\th(z\psi/2)}{z\psi D_m/\delta + k_2\th(z\psi/2)}.
\]
