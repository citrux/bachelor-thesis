\section{Воздействие СВЧ-поля на мембранный транспорт}

Рассмотрим действие СВЧ-поля на мембранный транспорт. Пусть в мембране
установился стационарный режим, при этом поле в мембране \( E^{(0)}(x) \).
Поместим теперь мембрану во внешнее поле \( E_m e^{i\omega t} \), причём
\( \abs{E_m} \ll E^{(0)} \). СВЧ-поле внесёт возмущение в распределение
ионов в мембране и, как следствие, в распеделение поля. Поэтому результирующее
поле можно представить в виде
\begin{equation}
    E = E_m e^{i\omega t} + E^{(0)} + E^{(1)} + \ldots
\end{equation}
Подставим это поле в полученное ранее уравнение \eqref{eq:epic-equation}:
\begin{gather*}
    \pcder{\left(E_m e^{i\omega t} + E^{(0)} + E^{(1)}\right)}{x}{t} =
     D\frac{\partial^3
         \left(E_m e^{i\omega t} + E^{(0)} + E^{(1)}\right)
    }{\partial x^3}
     -\\-
    \frac{z}{\abs{z}}u\pder{}{x}\left[
    \left(E_m e^{i\omega t} + E^{(0)} + E^{(1)}\right)
    \pder{\left(E_m e^{i\omega t} + E^{(0)} + E^{(1)}\right)}{x}\right].
\end{gather*}
Так как стационарное поле \( E^{(0)} \) так же удовлетворяет этому уравнению,
а \( E_m = \const \), то уравнение можно упростить, оставив при этом только
величины первого порядка малости:
\begin{gather}
    \pcder{E^{(1)}}{x}{t} = D\frac{\partial^3 E^{(1)}}{\partial x^3} +
    \frac{uzE^{(0)}}{\abs{z}}\ppder{E^{(1)}}{x} +
    2\frac{uz}{\abs{z}}\pder{E^{(0)}}{x}\pder{E^{(1)}}{x} + \nonumber \\
    \frac{uz}{\abs{z}}\ppder{E^{(0)}}{x} E^{(1)} +
    \frac{uz}{\abs{z}}\ppder{E^{(0)}}{x} E_m e^{i\omega t}.
\end{gather}
Это линейное дифференциальное уравнение в частных производных третьего порядка с
переменными коэффициентами.
