\begin{center}
    ВВЕДЕНИЕ
\end{center}
\addcontentsline{toc}{chapter}{Введение}
\vspace*{1cm}

\textbf{Актуальность работы.} Наличие высокочастотных полей влияет на процессы
    движения ионов, причем это влияние в большой степени зависит от уровня
    падающей мощности. Считается, что основное воздействие излучения низкой
    интенсивности связано с наличием резонанса внешних колебаний с частотой
    колебаний клетки, что приводит к активизации ее деятельности. Несмотря
    на большое количество работ в этом направлении
    \cite{bib:1,bib:2,bib:3,bib:4,bib:5,bib:6,bib:7}, вопросы, связанные с
    исследованиями физических механизмов транспорта ионов в результате
    воздействия сверхвысокочастотного электромагнитного излучения низкой
    интенсивности, все еще остаются. Наличие высокочастотных полей в
    настоящее время является постоянным фактором, поэтому изучение их
    влияния на живойорганизм необходимо. Таким образом, создание моделей,
    позволяющих описать этот процесс хотя бы с учетом ограничений и
    приближений, является актуальной задачей.

\textbf{Цель работы} заключается в получении уравнений, описывающих влияние
    высокочастотных возмущений на ионные токи в мембранах. При рассмотрении
    задачи ионного транспорта в электродиффузионном подходе используется
    уравнение Нернста—Планка
    \cite{bib:8,bib:9,bib:10,bib:11,bib:12,bib:13,bib:14,bib:15,bib:16,bib:17,bib:18},
    которое связывает ионный ток с распределениями концентраций и электрического
    потенциала в стационарном случае, то есть когда ток постоянен. Так как электрические
    колебания не стационарный процесс, то необходимо модифицировать уравнение
    для нестационарного случая. Возожность этого рассматривается в \cite{bib:8}.
    До сих пор попытки рассмотрения воздействия СВЧ-излучения на ионный
    транспорт опирались на стационарное уравнение \cite{bib:3, bib:19}.
    Однако рассмотрение нестационарного уравнения вместе с уравнением Пуассона
    позволяет достаточно строго подойти к определению воздействия внешнего
    СВЧ-поля на ионные токи.
