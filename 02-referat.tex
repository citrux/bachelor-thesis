\setcounter{page}{4}
\begin{center}
РЕФЕРАТ
\end{center}

Целью данной работы является получение уравнений, описывающих воздействие
СВЧ-излучения на ионные токи в клеточных мембранах. В ходе работы на основе
уравнений Нернста-Планка и Пуассона получены уравнения для нестационарных
процессов в мембранах. Применение этих уравнений для внешних периодических
возмущений позволило получить уравнения для поля в мембране, находящейся во
внешнем СВЧ-поле.
\vspace*{1cm}

\noindent Ключевые слова: мембранный транспорт, СВЧ-излучение,
электродиффузия, уравнение Нернста-Планка, уравнение Пуассона, теория
возмущений, нестационарный процесс.
\vspace*{1cm}

\begin{center}
    ABSTRACT
\end{center}

The purpose of this work is to obtain the equations describing the effects of
microwave radiation on the ionic currents in cell membranes. During the work
on the basis of the equation Nernst-Planck and Poisson equations are derived
for non-stationary processes in membranes. The use of these equations for the
external periodic perturbation yielded field equations in the membrane in an
external microwave field.
\vspace*{1cm}

\noindent Key words: membrane transport, microwaves, electrodiffusion,
Nernst-Planck equation, Poisson equation, perturbation theory,
non-stationary process.
