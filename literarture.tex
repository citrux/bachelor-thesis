\chapter{Электродиффузионные уравнения}
\section{Основные уравнения}

Если считать мембрану гомогенной средой, в которой может происходить диффузия и
миграция ионов, парциальный электрический ток ионов сорта к можно записать в
виде

\begin{equation}
    I_k = z_k u_k RT \left( \beta z_k c_k E - \pder{c_k}{x} \right),
    \label{eq:nernst-plank}
\end{equation}

где \( z_k \) -- заряд иона сорта к (в единицах заряда протона), \( u_k \) --
подвижность, связанная с коэффициентом диффузии \( D_k \) соотношением
\( D_k = u_k/\beta \), Е -- напряженность электрического поля.
Суммарная плотность ионного тока определяется как сумма
\[
    I = \sum_{k=1}^n I_k.
\]
В стационарном случае парциальные токи сохраняются, так что \( I_k = \const \),
и соотношение \eqref{eq:nernst-plank} представляет собой нелинейное
дифференциальное уравнение первого порядка, содержащее неизвестные функции
\( c_k \) и \( E \) и неизвестную постоянную \( I_k \). Это основное
электродиффузионное уравнение носит название уравнения Нернста -- Планка.
Если рассматриваемая система содержит \( n \) сортов ионов, то мы имеем \( n \)
уравнений \eqref{eq:nernst-plank} для \( n + 1 \) функций, в число которых
входят все \( c_k \) и \( Е \). Чтобы сделать задачу определенной, необходимо
располагать еще одним уравнением. Таким уравнением служит уравнение Пуассона.
Ввиду важности выпишем систему уравнений электродиффузионной задачи:
\begin{equation}
\left\{
    \begin{array}{l}
        \der{c_k}{x} - \beta z_k c_k E = - \frac{I_k}{z_k u_k RT},\\
        \der{E}{x} = \frac{F}{\eps}\sum_{k=1}^n z_k c_k.
    \end{array}
\right.
\label{eq:system_nernst-plank}
\end{equation}
Задавая \( 2n + 2 \) граничных условий, мы получаем окончательную постановку
проблемы. Кроме основной формы записи электродиффузионной задачи в виде
уравнений Нернста -- Планка и Пуассона \eqref{eq:system_nernst-plank} иногда
оказывается удобной запись посредством уравнения, из которого уже исключено
электрическое поле
\begin{equation}
    c_k\dder{c_k}{x} - \left(\der{c_k}{x}\right)^2 -
    \frac{I_k}{z_k D_k F}\der{c_k}{x} -
    \frac{F\beta}{\eps} z_k c_k^2 \sum z_i c_i = 0.
\end{equation}
В нестационарном случае необходимо воспользоваться уравнением непрерывности
\begin{equation}
    \pder{I_k}{x} + F z_k \pder{c_k}{t} = 0,
    \label{eq:continuity}
\end{equation}
которое совместно с \eqref{eq:nernst-plank} запишется в виде
\begin{equation}
    \pder{c_k}{t} =
        \frac{u_k}{\beta}\pder{}{x}\left(\pder{c_k}{x} - \beta z_k c_k E\right).
    \label{eq:nonstationery}
\end{equation}
Полный ток \( I_0 \) слагается теперь из тока смещения и ионного тока
\begin{equation}
    I_0 = \eps\pder{E}{t} + I,
\end{equation}
причем \( I_0 \) не зависит от координаты в мембране в противоположность своим
компонентам. Это легко увидеть из \eqref{eq:continuity}:
\[
    \pder{}{t}\left(\eps\pder{E}{t}\right) = F\sum_{k=1}^n z_k \pder{c_k}{t} =
    -\sum_{k=1}^n\pder{I_k}{x} = -\pder{I}{x},
\]
т. е.
\[
    \eps\pder{}{x}\left(\pder{E}{t}\right) = -\pder{I}{x}.
\]
Уравнение \eqref{eq:nonstationery} совместно с уравнением Пуассона из
\eqref{eq:system_nernst-plank}, дополненные начальными и граничными условиями,
полностью определяют нестационарную электродиффузионную задачу.

\section{Интегральная форма электродиффузионного уравнения}
Кроме дифференциальной формулировки электродиффузионных уравнений
\eqref{eq:nernst-plank} и \eqref{eq:system_nernst-plank} иногда оказывается
полезной интегральная [31]. Проинтегрируем уравнение Пуассона от некоторой точки
\( \overline{x} \) до \( х \), следуя работе Ропера [31]:
\begin{equation}
    E(x, t) = E(\overline{x}, t) + \frac{F}{\eps}\sum_{k=1}^n z_k
    \int_\overline{x}^x c_k(x', t) dx'.
    \label{eq:electric_field}
\end{equation}
Подставляя (1.80) в выражение для разности потенциалов на мембране
\begin{equation}
    \phi(t) = -\int_0^\delta E(x', t) dx',
\end{equation}
получаем
\begin{equation}
    \phi(t) = -\delta E(\overline{x}, t) - \frac{F}{\eps}\sum_{k=1}^n z_k
    \int_0^\delta \int_\overline{x}^x' c_k(x'', t) dx'' dx'.
\end{equation}
Подставим \(Е(\overline{x}, t)\) в \eqref{eq:electric_field}:
\begin{equation}
    E(x, t) = -\frac{\phi(t)}{\delta} -
    \frac{F}{\eps\delta}\sum_{k=1}^n z_k
    \int_0^\delta \int_\overline{x}^x' c_k(x'', t) dx'' dx' +
    \frac{F}{\eps}\sum_{k=1}^n z_k\int_\overline{x}^x c_k(x', t) dx'.
\end{equation}
Найдем теперь ток смещения
\begin{equation}
    \eps\pder{E}{t} = -\frac{\eps}{\delta}\der{\phi}{t} -
    \frac{F}{\delta}\sum_{k=1}^n z_k
    \int_0^\delta \int_\overline{x}^x' \pder{c_k(x'', t)}{t} dx'' dx' +
    F\sum_{k=1}^n z_k\int_\overline{x}^x \pder{c_k(x', t)}{t} dx'.
\end{equation}
Из условия непрерывности \eqref{eq:continuity}
\begin{equation}
    F\sum_{k=1}^n z_k\pder{c_k}{t} = -\pder{I}{x}.
\end{equation}
Следовательно,
\begin{gather}
    -\frac{\eps}{\delta}\der{\phi}{t} +
    \frac{1}{\delta}\int_0^\delta \int_\overline{x}^x' \pder{I}{x''} dx'' dx' -
    \int_\overline{x}^x \pder{I}{x'} dx' = \\ =
    -\frac{\eps}{\delta}\der{\phi}{t} +
    \frac{1}{\delta}\int_0^\delta [I(x', t) - I(\overline{x}, t)] dx' -
    [I(x, t) - I(\overline{x}, t)] = -\frac{\eps}{\delta}\der{\phi}{t} +
    \frac{1}{\delta}\int_0^\delta I(x', t) dx' - I(x, t).
\end{gather}
Суммарная плотность тока равна
\begin{equation}
    I_0(t) = -\frac{\eps}{\delta}\der{\phi}{t} +
        \frac{1}{\delta}\int_0^\delta I(x', t) dx'.
\end{equation}

Другое интересное соотношение получается, если разделить \eqref{eq:nernst-plank}
почленно на \( c_k(x) \) и проинтегрировать по всей толщине мембраны:
\begin{equation}
    \int_0^\delta \frac{l_k(x', t)}{c_k(x', t)} dx' =
    \beta z_k \int_0^\delta E(x', t) dx' - \int_{c_k(0)}^{c_k(\delta)} d\ln c_k=
    = -\beta z_k \phi(t) + \ln\frac{c_k(0)}{c_k(\delta)} =
    -\beta z_k[\phi(t) - \phi_k],
\end{equation}
где \( l_k = I_k / z_k u_k RT \), а \( \phi_k \)-- равновесный мембранный
потенциал для ионов сорта \( k \), определяемый соотношением Нернста
\[
    \phi_k = \frac{1}{z_k\beta}\ln\frac{c_k(0)}{c_k(\delta)}.
\]
В стационарном состоянии \( \phi(t) = \phi \) и \( l_k = \const \), так что
\begin{equation}
    I_k = z_k u_k RT l_k = -g_k(\phi - \phi_k),
    \label{eq:CVC}
\end{equation}
где парциальная проводимость \( g_k \) для ионов сорта \( k \) определена
формулой
\begin{equation}
    g_k = \frac{F z_k^2 u_k}{\int_0^\delta\frac{dx'}{c_k(x')}}.
    \label{eq:conductivity}
\end{equation}

Вообще говоря, \( g_k \) зависит от потенциала через концентрационный профиль
под знаком интеграла в \eqref{eq:conductivity}, так что вольтамперная
характеристика \eqref{eq:CVC} является нелинейной. Тем не менее запись
проводимости в форме \eqref{eq:CVC}, где выделен линейный по смещению потенциала
сомножитель, имеет определенный смысл. Во-первых, при вычислении проводимости в
пределе малого поля, как следует из \eqref{eq:conductivity}, можно использовать
равновесный концентрационный профиль, после чего решение сводится к одной
квадратуре. Во-вторых, при быстром изменении внешнего поля распределение
концентраций в мембране не успевает перестроиться, а поэтому на малых временах
формула \eqref{eq:CVC} выражает линейную зависимость парциального ионного тока
от сдвига потенциала относительно равновесного значения. Следует обратить
внимание на определенную аналогию между соотношением \eqref{eq:CVC} и
эмпирическими уравнениями Ходжкина -- Хаксли для ионных токов через биомембраны,
которые также имеют омический характер на малых временах [30].

\section{Приближенное решение Планка}
В точной формулировке \eqref{eq:system_nernst-plank} уравнения ионного
транспорта приводят к трудно обозримым результатам; поэтому в большинстве
случаев пользуются приближенными решениями, основанными на тех или иных
предположениях.

Начнем с краткого изложения приближения Планка [32], которое состоит в том, что
в мембране, по предположению, выполняется условие электронейтральности. Таким
образом, основная система уравнений электродиффузионной задачи
\eqref{eq:system_nernst-plank} сводится к приближенной системе
\begin{equation}
\left\{
    \begin{array}{l}
        \der{c_k}{x} - \beta z_k c_k E = - \frac{I_k}{z_k u_k RT},\\
        \sum_{k=1}^n z_k c_k = 0.
    \end{array}
\right.
\label{eq:system_nernst}
\end{equation}
в которую вместо уравнения Пуассона входит условие электронейтральности. Начало
координат совместим с левой границей мембраны. Граничные условия для
концентрации электролита в мембране зададим в виде
\begin{equation}
    \left.c_k\right|_{x=0} = c_k(0), \left.с_k\right|_{х=\delta} = c_k(\delta).
    \label{eq:boundary_conditions}
\end{equation}
Задача станет полностью определенной, если известны либо разность потенциалов,
приложенная к мембране \( \phi(0) - \phi(\delta) \), либо электрический ток.
В общем случае смеси электролитов сложного состава решение
\eqref{eq:system_nernst} связано со значительными трудностями. Поэтому, чтобы
выявить основные качественные особенности планковского приближения, рассмотрим
простейший пример бинарного электролита \( А^+В^- \), концентрации которого
слева и справа от мембраны различны. Воспользовавшись условием
электронейтральности, а затем складывая и вычитая почленно уравнения
Нернста -- Планка для анионов и катионов, сведем \eqref{eq:system_nernst} к
следующей системе
\begin{align}
    \der{c}{x} = \chi,              \label{eq:system_nernst_binary_1}
    c(x)\der{\phi}{x} = -\alpha,    \label{eq:system_nernst_binary_2}
\end{align}
где
\begin{equation}
    \chi = \frac{I_B u_A - I_A u_B}{2Ru_Au_B},\quad
    \alpha = \frac{I_B u_A + I_A u_B}{2Ru_Au_B\beta}.
    \label{eq:system_nernst_binary_subs}
\end{equation}
Отсюда следует, что распределение концентрации электролита в мембране в
планковском приближении характеризуется постоянным наклоном. Интегрируя
\eqref{eq:system_nernst_binary_1} с граничным условием
\eqref{eq:boundary_conditions}, получаем линейный концентрационный профиль
\begin{equation}
    с (х) = \chi х + с(0).
    \label{eq:nernst_binary_conc}
\end{equation}
Полагая \( х = \delta \), находим связь между параметром \( \chi \) и градиентом
концентрации:
\begin{equation}
    \chi = \frac{c(\delta)-c(0)}{\delta}.
    \label{eq:nernst_binary_chi}
\end{equation}
Проинтегрируем теперь уравнение \eqref{eq:system_nernst_binary_2},
воспользовавшись \eqref{eq:nernst_binary_conc}:
\begin{equation}
    \phi(х) = -\frac{\alpha}{\chi}\ln[\chi х + c(0)] + \const.
    \label{eq:nernst_binary_pot}
\end{equation}
С помощью \eqref{eq:nernst_binary_pot} находим разность потенциалов \( \phi \),
определенную как \( \phi(0) - \phi(\delta) \):
\begin{equation}
    \phi = \frac{\alpha}{\chi}\ln\frac{\chi\delta + c(0)}{c(0)}.
    \label{eq:nernst_binary_phi}
\end{equation}
Выразим здесь \( \chi \) с помощью \eqref{eq:nernst_binary_chi} и перепишем
\eqref{eq:nernst_binary_phi} в виде:
\begin{equation}
    \alpha = \frac{\phi[c(\delta) - c(0)]}{\delta\ln[c(\delta) / c(0)]}.
    \eqref{eq:nernst_binary_alpha}
\end{equation}
Из \eqref{eq:system_nernst_binary_subs} выразим \(I_A\) и \(I_B\) через
параметры \( \alpha \) и \( \chi \):
\begin{equation}
    \begin{array}{l}
        I_A = RТu_A (\alpha\beta - \chi),\\
        I_B = RTu_B (\chi - \alpha\beta),
    \end{array}
    \label{eq:nernst_binary_currents}
\end{equation}
а затем, подставив в \eqref{eq:nernst_binary_currents} выражения
\eqref{eq:nernst_binary_chi} и \eqref{eq:nernst_binary_alpha}, найдем ионные
токи
\begin{equation}
    I_k = RTu_k\frac{c(\delta) - c(0)}{\delta\ln[c(\delta)/c(0)]}\cdot
    \left[\psi - z_k\ln\frac{c(\delta)}{c(0)}\right],
    \label{eq:nernst_binary_currents_2}
\end{equation}
\[
    k = A, B,\quad z_A = 1,\quad z_B = -1.
\]
Полный электрический ток равен:
\begin{equation}
    I = I_A + I_B = \frac{RT[c(\delta) - c(0)]}{\delta\ln[c(\delta)/c(0)]}\cdot
    \left[(u_A + u_B)\psi + (u_B - u_A)\ln\frac{c(\delta)}{c(0)}\right].
    \label{eq:nernst_binary_current}
\end{equation}
Приравнивая электрический ток нулю, найдем диффузионный потенциал, который
устанавливается на мембране в условиях разомкнутой цепи:
\begin{equation}
    \psi_d = \frac{u_A - u_B}{u_A + u_B}\ln\frac{c(\delta)}{c(0)}.
    \label{eq:nernst_binary_psid}
\end{equation}
Последняя формула физически вполне прозрачна. Если анионы и катионы обладают
разными подвижностями, то на мембране возникает электрическое поле,
компенсирующее эту разницу. Разность потенциалов \eqref{eq:nernst_binary_psid},
конечно, не является термодинамически равновесной. Благодаря различию
концентраций слева и справа от мембраны через нее идет постоянный поток
электролита. Он отличен от нуля и при разомкнутой цепи, когда
\( \psi = \psi_d \). Но только в этом случае парциальные токи анионов и катионов
равны друг другу. Таким образом, диффузионный потенциал возникает как следствие
неравновесности системы, которая выражается в том, что \( c(\delta) \neq c(0) \)
при условии, что \( u_A \neq u_B \). Если подвижности анионов и катионов
совпадают, то присущая системе неравновесность электрически не проявляется --
диффузионный потенциал не возникает.

Как следует из \eqref{eq:nernst_binary_current}, электрический ток
пропорционален отклонению приложенной разности потенциалов от диффузионного
потенциала. Мембрана в планковском приближении -- омическая система. Линеен по
полю не только суммарный электрический ток, но и парциальные ионные токи. Этот
результат является следствием того факта, что в планковском приближении
концентрационный профиль, будучи линейным, не зависит от поля. Поэтому
проводимость \(g_k\) из \eqref{eq:conductivity} также не зависит от поля, а
ионный ток \( I_k \) оказывается линейной функцией сдвига потенциала
относительно парциального равновесного значения \( \phi_k \). Нетрудно заметить,
что \eqref{eq:nernst_binary_currents_2} имеет именно такую структуру. Чтобы
сделать это более наглядным, перепишем \eqref{eq:nernst_binary_currents_2}
в следующем виде:
\begin{equation}
    I_k = RTu_k\frac{c(\delta) - c(0)}{\delta\ln[c(\delta)/c(0)]}(\psi-\psi_k),
    \quad k = A,B.
\end{equation}

Каждый из ионных токов обращается в нуль только при таком значении внешнего
потенциала, которое совпадает с его равновесным потенциалом. Но ионы разных
зарядов имеют различные равновесные потенциалы. Так, нернстовские потенциалы
анионов и катионов прямо противоположны по знаку. Поэтому в системе может
установиться лишь стационарное состояние, зависящее от конкретных внешних
условий, в котором не реализуются парциальные ионные равновесия.

Согласно \eqref{eq:nernst_binary_pot}, потенциал в мембране изменяется по
логарифмическому закону. Это распределение характеризуется отличной от нуля
второй производной по координате, которая связана с плотностью заряда уравнением
Пуассона. Таким образом, исходя из условия электронейтральности, мы нашли
приближенное решение, которое приводит к выводу о наличии в системе отличного от
нуля объемного заряда. Малость этого заряда по сравнению с концентрацией ионов
служит критерием применимости условия электронейтральности. Планковское описание
ионного транспорта с успехом применяется в случае мембран, толщина которых много
больше размеров диффузных обкладок двойных слоев, находящихся в мембране у
границ раздела.

Планковское решение легко обобщается на случай бинарного электролита с ионами
произвольных зарядов, когда условие электронейтральности имеет вид
\( z_Ac_A + z_Bc_B = 0\). Суммарная плотность тока равна
\begin{equation}
    I = \frac{z_ART[c(\delta) - c(0)]}{\delta\ln[c(\delta)/c(0)]}\cdot
    \left[(z_Au_A - z_Bu_B)\psi + (u_B - u_A)\ln\frac{c(\delta)}{c(0)}\right].
\end{equation}
а диффузионный потенциал
\begin{equation}
    \psi_d = \frac{u_A - u_B}{z_Au_A - z_Bu_B}\ln\frac{c(\delta)}{c(0)}.
\end{equation}

В экспериментах часто встречается случай смеси двух электролитов с общим ионом,
например АВ и ХВ. При произвольных \( z_A, z_X, z_B \) задача является
нелинейной. Она линеаризуется в том случае, если какие-либо две зарядности
совпадают, например \( z_A = z_X = z \). Выражение для тока в этом случае имеет
вид:
\begin{gather}
    I = RT\frac{c_B(0) - c_B(\delta)}{\delta}\left\{
        z_Bu_B\left(1-\frac{z_B\psi}{\ln[c_B(\delta)/c_B(0)]}\right)
        + \right.\\ +\left.
        z\left[e^{-z\psi}[u_Ac_A(\delta) + u_Xc_X(\delta)] -
        [u_Ac_A(0) + u_Xc_X(0)]\right]
        \left(1-\frac{z\psi}{\ln[c_B(\delta)/c_B(0)]}\right)\right\}
        \label{eq:nernst_ternary_current}
\end{gather}

Диффузионный потенциал \( \psi_d \) получается из
\eqref{eq:nernst_ternary_current} при условии \( I = 0 \) как решение
трансцендентного уравнения. В справочных целях приведем решение для произвольной
смеси электролитов [31]. Пусть р -- число различных зарядностей,
\( \tilde{c}_k \) -- концентрация всех ионов зарядности \( z_k \),
\( с = \sum с_к \) -- суммарная концентрация. Тогда для парциального тока имеем
\begin{equation}
    I_k = z_ku_kRT\frac{c(0) - c(\delta)}{\delta}\cdot
    \frac{c_k(\delta)e^{-z_k\psi}-c_k(0)}
        {\tilde{c}_k(\delta)e^{-z_k\psi}-\tilde{c}_k(0)}\cdot
    \frac{\sum_{i=1}^{p-1}[1/z_k - f_i(\psi)]}
        {\prod_{i=1}_{(i \neq k)}(1/z_k - 1/z_i)},
\end{equation}
где функции \( f(\psi) \) удовлетворяют трансцендентному уравнению
\begin{equation}
    \exp(-\psi) = \left[\sum_{k=1}^p\frac{z_k\tilde{c}_k(0)}{f-1/z_k}\right]^f
    \cdot \left[\sum_{k=1}^p\frac{z_k\tilde{c}_k(\delta)}{f-1/z_k}\right]^{-f},
\end{equation}
имеющему \( р-1 \) решений. Отыскав ионные токи, можно найти суммарный
электрический ток и диффузионный потенциал.

\section{Гольдмановское приближение постоянного поля}
В случае тонких мембран, когда длина экранирования превосходит толщину мембраны, предположение электронейтральности теряет силу. Более убедительным является здесь приближение постоянного поля [33]. Точная система уравнений электродиффузионной задачи \eqref{eq:system_nernst-plank} в гольдмановском приближении имеет вид:
= - TTJ7F ’ £ = const. (1.110)
Граничными условиями, как и в предыдущем разделов, являются заданные на краях мембраны значения концентрации ионов ск (0) и си (S). Условие постоянства поля приводит к линеаризации уравнений Нернста -- Планка, которые теперь лег:ко интегрируются. Для ионного тока получается выражение
z\RTuk^ ск (0) -- ск (6) е
а концентрационный профиль имеет вид
zkEx
Ск {х) = (0) + [ск (б) -- ск (0)] ■
Здесь разность потенциалов на мембране ф попрежн ему определена как ф (0) -- ф (б), а Е = ф/б. В отличие от пдланковского
Рис. 6. Гольдмановские вольт- амперные характеристики для катионов
J -- с (0) > с (6);
2-- с (0) < с (б);
3-- с учетом кинетики поверхностных стадий
случая профиль концентраций теперь нелинеен по х ш зависит от поля. Если поле положительно, то на катионы (кротае концентрационного градиента) действует электрическая сила, направленная в ту же сторону, что и поле. Поэтому при си (0) > ск (б) профиль концентрации оказывается выпуклым. Прш обратном знаке поля профиль будет вогнутым. С увеличением абсолютной величины приложенного поля концентрация почти в о всех точках мембраны становится такой же, как на левой шли правой границе в зависимости от знака поля.
Зависимость ионных токов (1.111) от приложенной разности потенциалов в гольдмановском случае нелинейна. Только в симметричных условиях, когда концентрации ионов слева и справа от мембраны одинаковы, функция (1.111) становится линейной. Парциальные ионные токи обращаются в нуль при т аком значении внешнего потенциала, которое совпадает с парциальным нернстовским потенциалом. На рис. 6 изображены п арциальные вольтамперные характеристики для катионов в случгаях с (0) > с (б) (кривая 1) и с (0) < с (б) (кривая 2). При больших по аб- 
солютнои величине полях, когда концентрация практически во всей мембране постоянна, система ведет себя как омический проводник. Характер нелинейности вольтамперных кривых говорит о наличии у системы выпрямляющих свойств. Иными словами, при одинаковых по абсолютной величине, но разных по знаку сдвигах потенциала через мембрану протекают различные по абсолютной величине токи. В литературе этот эффект называется гольдмановским выпрямлением. Степень выпрямления, как это следует из (1.111), зависит от того, насколько сильно различаются концентрации ионов, заданные на границах мембраны.
В условиях разомкнутой цепи в системе возникает мембранный потенциал, который можно найти, приравняв нулю суммарный электрический ток. Выпишем результат для случая бинарного 1 : 1 электролита:
Т„ uaca(S)+ "всв(°) ,л iA^
“аса(°) + ил«в(6) ‘ ( }
Физическое содержание этой формулы такое же, как и \eqref{eq:nernst_binary_psid} для планковского потенциала. Гольдмановский потенциал возникает как следствие неравновесности системы. По своей структуре формула (1.113), конечно, отличается от \eqref{eq:nernst_binary_psid}, так как в основу ее вывода было положено условие постоянства поля, а не условие электронейтральности в каждой точке мембраны.
Если мембрана проницаема для ионов одного знака, причем заряды всех ионных компонентов одинаковы, то формулу типа (1.113) для мембранного потенциала можно получить, не требуя постоянства поля. Суммируя уравнения Нернста--Планка \eqref{eq:nernst-plank} по всем к, с учетом условия = 0 получим
Sz.u.
dty= • (1.114)
2W/c
Поскольку заряды всех ионов одинаковы, то из (1.114) получаем
dZukclt
zhu, с.
К а
ИЛИ
--rdln(2 м*с*)’
откуда непосредственно следует s Ук (5)
2 икск (0)
Гольдмановское приближение, в частности выражение для мембранного потенциала, широко используется при описании ионного транспорта через биологические мембраны. Однако фор-
мулы (1.113) и (1.115) непосредственно не могут быть применены для обработки экспериментального материала, так как в них входят неизвестные концентрации в мембране у ее границ, тогда как задаваемой величиной является концентрация ионов в объеме окружающих ее растворов. Кроме того, не следует забывать, что потенциал 'фс представляет только внутреннюю, собственно мембранную часть полной мембранной разности потенциалов, в которую входят еще равновесные поверхностные скачки. Считая, что мембрана не имеет фиксированного поверхностного заряда и ее толщина меньше длины экранирования, можно пренебречь поверхностными скачками потенциала, которые в этих предположениях малы, и выразить концентрации в мембране через объемные значения с помощью коэффициентов распределения. Тогда (1.113) приобретает вид
»-1.Мд+;д;и, (1-ив)
A Al “Г ГВСВ2
где введены проницаемости Рк = икчк!8 = уkDkjb. Формула (1.116), известная как соотношение Гольдмана -- Ходжкина -- Катца, определяет^ полный мембранный потенциал через концентрации ионов в растворах сц.
Если поверхностными скачками потенциала нельзя пренебречь, то полный мембранный потенциал можно представить в виде суммы
■ф = -фх + г|)2 + 1|з3. (1.117)
Величина i|)2 определяется формулой (1.115), а ^ и TJJ3 можно связать с концентрациями и коэффициентами распределения:
fc-4-b.^Sil. (1.118)
2 г ‘/и
Подставляя в (1.117) выражения (1.115) и (1.118) и суммируя, получим
t4hT77T' (1Л19>
При выводе этой формулы, которая связывает полный трансмембранный скачок потенциала с объемными концентрациями электролита, предполагалось, что мембрана проницаема только для ионов одного знака. Оправдать «запирание» мембраны для ионов противоположного знака можно в том случае, если она несет значительный фиксированный заряд, вытесняющий из мембраны коионы, как следует из (1.47).
Учет поверхностной стадии. Приближение постоянного поля в изложенной выше форме предполагает наличие равновесного распределения ионов на границе мембраны с раствором, т. е. лимитирующая стадия ионного транспорта тем самым автомати-
чески перемещается в объем мембраны. Однако это допущение отнюдь не является критическим и его легко избежать. Предположим, что мембрана проницаема для ионов только одного типа, которые входят в мембрану из раствора с константой скорости kt и выходят из нее с константой к2. Величины кг и /с2, вообще говоря, могут зависеть от приложенного потенциала. Это обстоятельство будет специально рассмотрено в главе 2. Решение уравнений ионного транспорта (1.110) в приближении постоянного поля дает следующую вольтамперную характеристику:
I = zFk -1Й~ С2-   . (1.120)
,2ф
т
Графически эта зависимость представлена на рис. 6, кривая 3. Полученная вольтамперная характеристика, в отличие от гольд- мановской, имеет насыщение. Предельные токи равны соответственно
1г = zFk^', /2 = -- zFkxCi. (1.121)
В эти выражения входят только характеристики границы. Увеличивая разность потенциалов на мембране, мы ускоряем объемную стадию переноса ионов и в конце концов она становится настолько быстрой, что перестает лимитировать мембранный транспорт. В результате член, пропорциональный DmJб, исчезает из формулы для вольтамперной характеристики. Любопытно сравнить выписанные выше предельные токи с аналогичными выражениями (1.70), полученными для неперемешиваемых слоев. По структуре формулы очень похожи, а при замене Dth -->- кх они просто переходят друг в друга. Таким образом, роль границ мембраны с раствором может быть вполне аналогичной роли неперемешиваемых слоев. Это обстоятельство будет использовано в модели переносчиков.
Отличие вольтамперной характеристики (1.120) от гольдма- новской состоит еще и в том, что здесь уже нельзя ввести понятие проницаемости Р как множителя, содержащего константы скоростей переноса и определяющего амплитуду вольтамперной характеристики. Величины Dmlб, кг и к2 входят в более сложных комбинациях с функциями потенциала. Формула (1.120) является более общей. Гольдмановская вольтамперная характеристика получается из нее как частный случай, если скорости граничных переходов кг и кг устремить к бесконечности, оставляя их отношение, равное коэффициенту распределения kjk2= у, конечным.
В симметричных системах вольтамперная характеристика (1.120) также становится симметричной:
z*FDmklCxр tbWI2) (, ,т
S ' *1|>о /6 + ЫМ**/2) ' V
Отсюда легко получить предельную формулу для случая, когда внутри мембраны подвижность ионов очень высока, а все сопротивление сосредоточено на границе:
I = zFh\c th (гт|)/2). (1.123)
Этот случай прямо противоположен гольдмановскому.
В заключение выпишем еще формулу для проводимости мембраны в предельном случае малого внешнего поля. Предельный переход в вольтамперной характеристике (1.120) дает:
zWc
2/*1 + а/т1>т ■ (1.124)
Проводимость прямо пропорциональна концентрации проникающего иона в растворе с, а слагаемые в знаменателе описывают сопротивления последовательных стадий, через которые проходит ион. Первый член описывает границы, а второй -- объем мембраны; однако он не равен просто 8/Dm, а поделен еще на коэффициент распределения. Причина вполне понятна: чем выше коэффициент распределения ионов между мембраной и раствором, тем меньше сопротивление объема мембраны.
Принцип независимости потоков. Кроме электрических характеристик, обсуждавшихся выше, важное значение имеют односторонние потоки ионов jh и fk, которые измеряются с помощью меченых частиц. Экспериментальные исследования ионного транспорта через биологические мембраны показали, что в целом ряде случаев односторонние потоки оказываются независимыми. Этот факт ввиду его общности и важности получил название принципа независимости.
В рамках общей электродиффузионной задачи \eqref{eq:system_nernst-plank} анализировать вопрос о связи односторонних потоков не представляется возможным, так как не удается получить обозримых аналитических выражений для соответствующих величин. Поэтому обратимся к приближению постоянного поля, которое наиболее адекватно описывает перенос заряженных частиц через тонкие мембраны. Выражения для односторонних потоков можно получить, полагая в (1.111) ck (б) и ск (0) попеременно равными нулю и разделив Jk на zkF:
Г V^c*(°) г иокч
[36 (1 -- в *ф) 36 (1 -- е И
Направленный слева направо поток jk мы будем называть выходящим, а справа налево -- входящим, ассоциируя тем самым левый раствор с внутренней частью клетки, а правый--с наружной. Ионный ток 1к, очевидно, равен
h = zkF (Jk -- Тк). (1.126) 
Выходящий поток jh линейно зависит от концентрации слева от мембраны, но не зависит от концентрации справа. Напротив, входящий поток зависит только от концентрации справа. Таким образом, изменяя, например, концентрацию слева от мембраны, мы воздействуем на выходящий поток, в то время как входящий поток остается неизменным, если разность потенциалов поддерживается постоянной. Это означает, что входящий и выходящий потоки в гольдмановском приближении являются независимыми. Экспериментальная проверка принципа независимости в случае биологических мембран имеет принципиальное значение, так как это решает вопрос о справедливости предпосылок теории.
Составим с помощью (1.125) отношение односторонних потоков:
1±. = (0) eV =
Тк мб)
Эта формула, называемая соотношением Уссинга [34], также имеет важнейшее значение для проверки основ электродиффузион- ной теории. Отметим, что односторонние потоки, удовлетворяющие соотношению Уссинга, могут не подчиняться принципу независимости.
Формула Уссинга может быть получена из общих электродиф- фузионных уравнений без предположения о постоянстве электрического поля. Чтобы показать это, удобно представить парциальный ионный поток в виде
h = D^JL(ckezK% (1.128)
Рассмотрим случай, когда слева от мембраны находится раствор, содержащий проникающие ионы А, а справа -- раствор, в ко- торомприсутствуют проникающие ионы В. Предположим, что DA = --DB, ZA = ZB• Будем считать заданными концентрации на краях мембраны СА (0) и св (б), тогда как св (0) = 0, сА (б) = 0. Подставим в очевидное тождество J'A]B = ]BJA выражение (1.128) для потоков
и, воспользовавшись постоянством потоков, внесем их под знаки дифференцирования
-3rVAcBeI*)=-±r{jBCAe*). (1-129)
В силу принятых граничных условий, интегрируя (1.129),получаем
]АСВ (б) е2ф(5> = -- ]ВСА (0) е2<М°>,
откуда следует 1А СА (°) 
что является обобщением (1.127). Этот вывод можно провести и для случая трехмерной диффузии [35].
1.7. Эффекты двойных электрических слоев
Мембрана без поверхностного заряда. Недостатки приближения постоянного поля хорошо известны и неоднократно обсуждались в литературе. В этом приближении, как правило, полностью


Рис. 7. Распределение потенциала с учетом поляризации двойных электрических слоев в растворах Пояснение в тексте
 9

 <st| 0  х
~£/г   \8/г




игнорируется роль раствора, омывающего мембрану, не учитывается падение потенциала и изменение концентрации в двойных электрических слоях. В действительности решение задачи об ионном транспорте в самой мембране должно быть сопряжено с решением такой же задачи в омывающих мембрану растворах. Этот анализ был проведен в работах [36], а затем с учетом некоторых упрощающих предположений в [37--411.
Следуя Лойгеру с сотр. [39], рассмотрим влияние двойных электрических слоев у границ раздела мембраны с водными растворами на прохождение тока через мембрану. Пусть в растворе содержится полностью диссоциированный бинарный электролит А+В“ концентрации с, который в некоторой степени растворим в фазе мембраны, и фоновый электролит X+Y~ концентрации с*. Будем предполагать, что на границах имеет место равновесие, а толщина мембраны значительно меньше длины экранирования, так что применимо приближение постоянного поля. Вначале рассмотрим случай, когда мембрана не несет собственного фиксированного заряда. Тогда в отсутствие внешнего поля симметричная система раствор--мембрана -- раствор будет эквипотенциальной по координате х. При наложении внешней разности потенциалов г|) некоторая ее доля j +я|з3 будет падать в растворах, в области диффузных обкяадок двойных электрических слоев, а часть -- на самой мембране (рис. 7). В отсутствие электрического тока двойные электрические слои описываются равновесными формулами (1.21). Учитывая высокое сопротивление мембран в интересующих нас случаях, естественно предположить, что соотношения (1.21) остаются справедливыми и при протекании слабого тока. В приближении постоянного поля в мембране справедливы уравнения (1.110), которые легко интегрируются. Распределение
концентраций в мембране имеет вид:
С А (ж) = гАС (Р -- qe^x&),
(1.131)
С в (х) = ХвС (р -- qe~^x‘&), а парциальные ионные потоки равны
1А = TACDAP^/Ь, ]В = -- TBcDBp\Ji2/6, (1.132)
где
р = sh -у- j sh -у-, q = sh - j sh 4^-.
Из условия непрерывности нормальной компоненты электрической индукции получаем трансцендентное уравнение, определяющее ар2 через г|):
Суммарная плотность тока равна
1 = 8Ф = ^ (ТлВл + UDB) *, , (1.134)
где g -- проводимость. Это равенство может служить определением g (т|з) при произвольных напряжениях. При малых полях имеем
1 = £оФ = -у- (TADA + ГBDb) ф. (1.135)
Эта формула совпадает с обычным гольдмановским выражением в симметричной системе, где вольтамперная характеристика линейна. Отклонение от линейности за счет эффектов двойного слоя удобно характеризовать отношением
J_=^£b(jW2|_ 113
go i|! sh (т|га/2) v '
Как следует из формул (1.134)--(1.136), наличие двойных слоев в растворах оказывает двоякое влияние на проводимость мембраны. Прежде всего собственно на мембране теперь падает не весь приложенный потенциал T|J, а только его часть г|з2. Это уменьшает проводимость. Но одновременно, поляризуя двойные свои в растворах, внешний потенциал повышает концентрацию носителей в мембране, как это видно на рис. 8, и тем самым увеличивает проводимость. Последний эффект более сильный, так что проводимость в конечном счете растет. На рис. 9 показан рост проводимости с полем, который заметен при концентрациях меньше 10_3 М. В более концентрированных растворах этот эффект мал.
Мембрана с поверхностным зарядом. Задача об отыскании вОльтамперной характеристики мембраны при наличии фиксиро- 
ванного заряда решается аналогично [41]. Предполагается, что на боковых поверхностях мембраны имеется поверхностный заряд с плотностью а' и а" соответственно. Диффузные обкладки двойных электрических слоев со стороны растворов описываются равновесными соотношениями, а перенос ионов через мембрану рассматривается в приближении постоянного поля. Не останавливаясь на деталях стандартного вывода, приведем окончательные формулы:


где с -- концентрация ионов в объеме раствора, а остальные обозначения ясны из рис. 10. Потенциалы на боковых поверхностях


св/с, св1!ГвС С я / с
сд /и 1 1 / \
СЯ /Уя с 1 ■ св /с 1 1
-Ч- -2-1/2 0 //2 2 UX/S
Рис. 8. Профили концентраций при г|) = 4, ет =2 е = 78,5, 6 = 70 А, х-* = 96 А (с = 10-» М) [39]




Рис. 9. Проводимость как функция напряжения я]з, вычисленная по (1.134) [39]
е = 2; концентрация (в M): 1 -- 10-1, 2 -- 10~г, 3 -- 10--3, i --10-*
мембраны и г|/' находятся из системы трансцендентных уравнений:
(1.138)
_!»_ w _ V) = sh (_|L + ,
где о0 = ех/2лр. Разбив (1.137) на два слагаемых, удобно ввести парциальные проводимости dl.
goi^lim-r1-, г = А, В,
ср--*0 Ф
которые при а' = а" = а оказываются равными
goi = goi|-=o-e'^, i[)0 = lim (1[Л гр"), zi=±i. (1.139)
Ф--*о 
Как следует из (1.137), поверхностный заряд определенного знака приводит к уменьшению парциального тока коионов и увеличению тока противоионов. Этот эффект непосредственно связан с наличием диффузных обкладок двойных слоев в растворах -- поверхностный заряд вызывает повышение концентрации противоионов как у поверхности, так и в самой мембране, поскольку межфазное распределение частиц предполагается равновесным. В концентрированных растворах влияние поверхностного заряда
Рис. 10. Распределение электрического потенциала при наложении внешнего поля на мембрану с поверхностным зарядом [41]
а' велико. и положительно, а а" мало и отрицательно
выражено слабо, так как потенциал tp0, определяемый в этой области соотношением •vj),, = 2ale0, стремится к нулю с ростом концентрации электролита.
ванная плотность тока как функция безразмерного потенциала для случая а' = -- а” и с = 10~2 М. Влияние концентрации электролита иллюстрируется на рис. 12.
1.8. Влияние объемного заряда
Случай малого внешнего поля. До сих пор мы пренебрегали влиянием объемного заряда внутри мембраны. В случае хорошо растворимых в липидной фазе ионов эти эффекты могут стать существенными. Проводимость мембраны в малых полях с учетом объемного заряда можно найти [40] на основе решения уравнения Пуассона -- Больцмана (1.49). Предположим, что в мембране растворимы ионы только одного сорта с зарядом z. Обкладки двойных слоев в растворах описываются равновесной формулой (1.21). Характер распределения потенциала ясен из рис. 13. Вводя безразмерные переменные
I = 2x16, I = -- яб3/|3/2ет D,
получим из общей электродиффузионной системы замкнутое уравнение для гр:
dl3 ' дф dl " ’ U.-14U)
которое можно один раз проинтегрировать
dhp_ , /_£Ф.\а _ _ Jt , А
ах? + 2 \ di ) ё + ’
где А -- постоянная интегрирования.
Граничные условия и непрерывность потенциала и его производных делают задачу полностью определенной. В состоянии равновесия, когда I = 0, имеем
ij> (£) = ip (0) + 2z In | cos (V4xm6ai) |, (1.141)
a = е-2ф(0)/2)
что является обобщением (1.51) на случай произвольных потенциалов. Чтобы не возникло недоразумений, укажем, что малой величиной в данной задаче считается приложенная разность потенциалов, тогда как изменение вдоль х равновесного внутреннего потенциала яр при 1 = 0 может быть велико по сравнению с единицей.
Имея равновесное распределение потенциала (1.141), нетрудно найти концентрационный профиль в мембране при 1 = 0:
с (|) = yca2/cos2 (xm8a|/4). (1.142)
Проводимость в малом поле, согласно \eqref{eq:conductivity}, выражается через
распределение концентрации:
+ i
1 _ 6 (* dt
go ~ 2DF ) с (I) ’
-- l
или окончательно


Если в мембране мало носителей, т. е. хт6 --0, то a 1, г|э (0) « ж 0; g0 да рГОус/б, т. е. проводимость, вычисленная с учетом объемного заряда, совпадает с найденной в приближении постоянного

Рис. 13. Схематическое распределение потенциала в мембране и окружающих растворах при наложении внешнего поля
-/z 0 /г х

Рис. 14. Проводимость мембраны в слабом поле, отнесенная к проводимости, полученной в приближении постоянного поля как функция y.„fi [40]
При расчетах по (1Л43) использовались значения *-* < 100 А (с>10-3 М); £?п = 2 е = 78,5; 6 = 70 А
поля. При произвольных хт б такое совпадение не имеет места. На рис. 14 построена зависимость отношения проводимости g0 из (1.143) к гольдмановской проводимости (1.135) при малом поле £о(об.зар.)/£о(пост.поле) от хт8, которая прекрасно иллюстрирует область применимости приближения постоянного поля. Вплоть до xm8 -- 1, т. е. ус ~ 5-10~5 М гольдмановское приближение справедливо с точностью до нескольких процентов. Эта оценка совпадает с полученной ранее из равновесных формул для потенциала в мембране.
При постоянной концентрации электролита в растворе, увеличивая только коэффициент распределения у, можно перейти к другому предельному случаю, когда в мембране много носителей, так что х,т8 1. Для проводимости в этом случае получаем:
g 0(7^00)= mmD/6s.
Таким образом, при неограниченном возрастании коэффициента распределения проникающего иона проводимость стремится к конечному пределу. Соответственно концентрация в центре мембра- 
ны также ограничена при у ->• оо и равна с(0) == яет/2Ед^.
Случай, внешнего поля произвольной величины. В общем случае, когда внешнее поле не мало, удобно обезразмерить систему \eqref{eq:system_nernst-plank} несколько иным способом, а именно:
х = s6, zc = пс (0), Е -- ё/zfib,
а = AnfiF&c (0) /в„, I = l'DFc (0) /б,
где с (0) -- концентрация на левой границе, z -- заряд проникающего иона, s, /г, Е, I* -- безразмерные длина, концентрация, напряженность электрического поля в мембране и плотность тока, « -- квадрат отношения толщины мембраны к толщине дебаев- ского слоя. Начало координат поместим на левой границе мембраны. Переписанная в этих безразмерных величинах система \eqref{eq:system_nernst-plank} примет вид


Как обычно, будем считать заданными концентрации на границах мембраны. Выражение
(1.147)
где А -- некоторая константа, является первым интегралом. Уравнение для <§ имеет вид:
т_
ds
-- а Гз + Аа.



где £ = zs/2 -- Azl2I*, (у) -- цилиндрическая функция по
рядка v от аргумента у, i -- мнимая единица. Тогда концентрация в мембране запишется как


72
L-*u
72
z 55
I s А,
/ z Az \*/«
4 i


Это неявное выражение для вольтамперной характеристики содержит неизвестные постоянные, которые находятся из граничных условий. Численный анализ подобных вольтамперных характеристик был проведен в работе [42], а в [40] на ЭВМ решалась
Рис. 15. Нормированная плотность тока I как функция напряжения при v=l, с=0,1 М, ет = 2, б = 100А [40]
Концентрация| проникающего иона (в М):
1 -- ю-*,
2 -- 10-*, 3 -- ю~а.
Пунктир соответствует предельной проводимости при больших концентрациях
непосредственно система \eqref{eq:system_nernst-plank} х. Полученные там результаты показывают, что вольтамперные характеристики остаются практически линейными во всей области параметров, представляющей интерес (рис. 15). Таким образом объемный заряд, не изменяя формы вольтамперных характеристик, оказывает влияние на их наклон.
1.9. Нелинейные вольтамперные характеристики
В обсуждавшихся выше случаях токи росли с потенциалом сравнительно медленно: либо линейно, либо еще медленнее. В эксперименте же нередко наблюдаются довольно круто поднимающиеся кривые. Поискам причин такого поведения был посвящен ряд работ [43, 35, 39--44]. Ниже мы рассмотрим несколько возможных механизмов.
Неоднородность потенциальной энергии иона в мембране. Изучая ионные потоки в мембране, мы до сих пор считали мембрану однородной. Иными словами, стандартный химический потенциал иона (д,0 до сих пор считался независящим от положения внутри мембраны. В общем случае это может быть не так. Причиной могут быть как силы изображения, рассмотренные в разделе 1.3, так и неоднородность самой мембраны. Не вдаваясь в дальнейшее обсуждение причин зависимости стандартного химического потенциала от координаты, рассмотрим, как эта зависимость отражается на вольтамперной характеристике мембран.
Зависящую от координаты часть стандартного химического потенциала Ц°(я) проникающих ионов в мембране обозначим W (х). Поскольку разделение стандартного химиче-
1 Интересный метод решения системы \eqref{eq:system_nernst-plank} был разработан в [43] при исследовании прохождения тока через область диффузного двойного слоя на границе металл -- раствор. 
ского потенциала на постоянную и переменную части несколько произвольно, потребуем, чтобы функция W (х) обращалась в нуль на правом краю мембраны: W (б) = 0. Кроме того, как и выше, будем пользоваться безразмерной энергией w (х). Ионные токи и в этом случае описываются формулой \eqref{eq:nernst-plank}, но с дополнительным членом, пропорциональным градиенту стандартного химического потенциала. Будем считать, что концентрация ионов внутри мембраны мала и применимо приближение постоянного поля. Здесь необходима оговорка. Постоянным будет только внешнее электрическое поле, т. е. внешний потенциал, наложенный на мембрану, будет изменяться линейно, а полный электрохимический потенциал ведет себя гораздо более сложным образом. В такой постановке задача решается без труда, и мы получаем вольтамперную характеристику:
/ -- zFky
5

о

Нетрудно видеть, что эта формула является обобщением вольт- амперной характеристики (1.120). Так же как и в (1.120), здесь учитывается кинетика ионных переходов на границе, но кроме того принят во внимание переменный профиль потенциальной энергии внутри мембраны. Если здесь положить w (х) = 0, то мы сразу же в качестве частного случая получим формулу (1.120).
Для удобства дальнейшего анализа рассмотрим симметричные системы с концентрацией проникающего иона с и предположим, что на границе мембраны с раствором поддерживается равновесное распределение ионов. На математическом языке это означает, что константы кг и /с2 устремляются к бесконечности, а их отношение остается равным коэффициенту распределения у. Тогда вольтам- перная характеристика принимает вид:
/ = zFDyc у

о

Стоящий в знаменателе интеграл представляет собой аналог сопротивления мембраны. Его значение определяется формой потенциальной энергии и> (х). Наибольший вклад в этот интеграл дают области, где потенциальная энергия велика, т. е. области потенциальных барьеров. Остальные участки мало существенны. Если внутри мембраны имеется один или несколько резко выделяющихся потенциальных барьеров, то вычисление интеграла в выражении (1.153) упрощается. Рассмотрим несколько примеров.
Пусть в мембране имеются два потенциальных барьера, расположенных у самых ее границ (рис. 16, а). Поскольку переход че- 

8 х
Рис. 16. Различные варианты энергетически неоднородной мембраны

Пояснение в тексте
рез них будет затруднен, то этот случай совершенно аналогичен учету кинетики ионных переходов на границе мембраны. Если барьер достаточно узкий, то в интеграле, стоящем в формуле (1.153), все члены, не содержащие w(x), можно вынести за знак интеграла. Оставшийся интеграл по одному барьеру обозначим через
Н = ^ ew(x) fa' (1.154)
Тогда вольтамперная характеристика принимает вид
/==^gr(1.155)
Эта функция совпадает с вольтамперной характеристикой (1.123), выписанной для случая, когда лимитирующей стадией является перенос ионов через границу мембраны. Вольтамперная характеристика в этом случае поднимается медленнее линейной и выходит на насыщение.
Рассмотрим другой случай. Предположим, что в мембране имеется один узкий барьер посередине (рис. 16, б). Проводя выкладки, аналогичные предыдущим, легко находим
I = (IzFDyciH) ?h (zip/2). (1.156)
Эта вольтамперная характеристика растет круче линейной. При больших потенциалах она растет, как exp (zip/2).
Предположим, наконец, что в мембране имеются два потенциальных барьера (рис. 16, е), каждый из которых расположен на рас- 
стоянии 6/4 от края. Тогда вычисления дают I = (2zFDyc/H) ?li(zi|-/'4).
В заключение рассмотрим еще один пример. Выше было показано, что при учете сил изображения потенциальная энергия вблизи границ имеет две довольно глубокие ямы. Аппроксимируем эту потенциальную энергию кривой, показанной на рис. 16, 8. Подстановка этой функции в интеграл (1.153) дает вольтампер- ную характеристику
__ z2FD"fc\|) егф -- 1
(1.158)
6еи
Степень нелинейности определяется отношением проводимости при произвольном напряжении g к предельной проводимости в малых полях g0. Вычисляя эти величины по формуле (1.158), получаем при больших потенциалах
g/g0~(l + 2£/6)exp(zi|)|/6). (1.159)
Таким образом, вольтамперная характеристика растет экспоненциально, однако показатель роста не слишком велик, поскольку он определяется отношением ширины ямы к толщине мембраны %/8.
Эффект Вина. Нелинейность вольтамперной характеристики может быть вызвана и сдвигом ионного равновесия «диссоциация- рекомбинация» в мембране под действием электрического поля (эффект Вина) [44].
Пусть в фазе мембраны находятся ионы А+ и В-, которые в основном ассоциированы в ионные пары А+В~. В отсутствие электрического поля имеет место равновесие
А+ + В- ^ А+В~
с константой диссоциации Cm = ^d/^Л = СаСв/сдв,
где сАв -- концентрация ионных пар в мембране в равновесии. Как показал Онсагер, константа ко зависит от электрического поля, a kR остается неизменной:
kD (Е) /кв (0) = 1г (2 УЩШЩ//ера|£|/е, (1.160)
где Ij -- модифицированная функция Бесселя первого рода. В приближении постоянного поля имеем
Ы£)/Ы0) = /№,
где -- разность потенциалов на мембране. 
Диффузионномиграционные потоки подчиняются уравнению непрерывности
~ШГ = ~ЗГ = кп°АВ ~ kR°ACB’ (1.161)
kDcAB -- квсАсв = kRcl [/ (ф) -- cAcB/cl],
где с0 = сА = св. Здесь использовано предположение, что ионных пар значительно больше, чем свободных ионов, так что наложение внешнего поля не влияет на их концентрацию. Переходя к безразмерным переменным
I = 2х/Ь, у = (сА + св) /2с0, z -- {сА св) /2с0,
получаем систему уравнений
1г+4-|-+^^^+22-^ = 0’ (1Л62)
2 dz _ Rf
У ~ ~R~ ’
где
А = с0б2/сй/4О, Д, = 6/2 c0FD,
a R -- интегральное сопротивление мембраны на единицу поверхности, которое определяется соотношением
I = F(jA -- /в) = ^/ДР-
В качестве граничных условий задается простейшая кинетика перехода ионов через поверхность раздела. Например, на правой границе раздела (| = 1) имеем
U = кгАсА (1) -- klAcs, )В = k2BcB (1) -- klBcs, (1.163)
где cs -- концентрация в растворе. Будем считать, что константы к одинаковы для обоих ионов и не зависят от поля. Выражая поверхностную реакцию через плотность тока, найдем граничные условия для %:
z (1) = дая, (1.164)
где
Rb = I/2c0Ffik%.
Нетрудно показать, что при малом поле сопротивление мембраны имеет вид
R0 = Rf + 2Rb, (1.165)
где Rb -- сопротивление поверхности. При произвольном поле решение можно получить в том случае, когда А 1 и Rf Rb:
ir:~ ■T1111W + AW■ Tlh (i) {'№) - ~t*\ ’ (1 '166)
2
где
go = Vi?0.
о г и 6 у О Z И-
Рис. 17. Проводимость как функция напряжения, полученная численным интегрированием (1.162) [44]
a: RyRj = Ю2; значения А: 1 -- 10~*, 2 -- 2-10~4, 3 -- 5-10-4, 4 -- 10~3; б; значения Rb/Rf'. 1--20, 2 -- 50, 3 -- 100, 4 -- 200; А = 2-10--*
На рис. 17, а представлена функция g/g0 при разных значениях А, а на рис. 17, б -- g/g0 при разных значениях Rb/Rj. Эффект Вина приводит к заметной нелинейности только при больших Rb/Rf. Это связано с тем, что скорость возникновения дополнительных свободных носителей мала по сравнению со скоростью их диффузии (А 1), так что концентрация зарядов в мембране остается практически неизменной, если только высокий поверхностный барьер не способствует их накоплению.
