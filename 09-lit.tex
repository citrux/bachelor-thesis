\addcontentsline{toc}{chapter}{Список использованных источников}
\def\bibname{СПИСОК ИСПОЛЬЗОВАННЫХ ИСТОЧНИКОВ}
\begin{thebibliography}{99}
\bibitem{bib:1} Шеин, А. Г. Математические модели влияния СВЧ-излучения низкой
    интенсивности на пассивный транспорт ионов : монография / А. Г. Шеин, Н. В.
    Грецова ; ВолгГТУ.~--- Волгоград : ИУНЛ ВолгГТУ, 2012.~--- 148 c.
\bibitem{bib:2} The influence of microwave radiation from cellular phone on
    fetal rat brain~/ J. Jing [et al] // Electromagnetic Biology and
    Medicine.~--- 2012.~--- Vol. 31, № 1~--- P.~57--66.
\bibitem{bib:3} Шеин, А. Г. Токи через мембрану с учетом наличия высокочастотных
    составляющих / А. Г. Шеин, Д. А. Барышев // Биомедицинская радиоэлектроника.~---
    2009.~--- №~4.~--- С. 4--9.
\bibitem{bib:4} Warchalewski, J. R. Influence of microwave
    heating on biological activities and electrophoretic pattern of
    albumin fraction of wheat grain /  J. R. Warchalewski, J. Gralik //
    СerealСhemistry.~--- 2010.~--- Vol. 87, № 1.~--- P.~35--41.
\bibitem{bib:5} Исмаилов, Э. Ш. Биофизическое действие СВЧ-излучений / Э. Ш.
    Исмаилов.~--- Москва : Энергоатомиздат, 1987.~--- 144 с.
\bibitem{bib:6} Тамбиев, А. Х. Влияние КВЧ-излучения на транспортные свойства
    мембран у фотосинтезирующих организмов / А. Х. Тамбиев, Н. Н. Кирикова,
    Е. Н. Маркарова // Биомедицинская радиоэлектроника.~--- 1997.~--- №~4.~--- С.
    67--76.
\bibitem{bib:7} Чопчиян, А. С. О краевых задачах для уравнений Нернста--Планка
    и Пуассона / А. С. Чопчиян, Е. Н. Коржов // Системы управления и
    информационные технологии.~--- 2009.~--- №~38.~--- С. 200--203.
\bibitem{bib:8} Маркин, В. С. Индуцированный ионный транспорт / В. С. Маркин,
    Ю. А. Чизмаджев.~--- Москва : Наука, 1974.~--- 252 с.
\bibitem{bib:9} Рубин, А. Б. Биофизика. В 2 т. Т. 2. Биофизика клеточных
    процессов / А. Б. Рубин.~--- Москва : Высшая школа, 1999.~--- 464 с.
\bibitem{bib:10} Биофизика : учебник для студентов вузов / В. Ф. Антонов [и др.] ;
    под ред. Антонова.~--- 3-е изд., испр. и доп.~--- Москва : Владос, 2006.~--- 287 с.
\bibitem{bib:11} Hille, B. Ion channels of excitable membranes / B.
    Hille.~--- 3rd ed.~--- Sunderland : Sinauer Associates, 2001.~--- 814 p.
\bibitem{bib:12} Плонси, Р. Биоэлектричество: Количественный подход / Р. Плонси,
    Р. Барр.~--- Москва : Мир, 1992.~--- 366 c.
\bibitem{bib:13} Биофизика / П. Г. Костюк [и др.] ; под ред. П. Г. Костюка.~--- Киев: Высшая
    школа, 1988.~--- 504 с.
\bibitem{bib:14} Иваницкий, Г. Р. Математическая биофизика клетки / Г. Р.
    Иваницкий, В. И. Кринский, Е. Е. Сельков.~--- Москва : Наука, 1987.~---
    308 c.
\bibitem{bib:15} Лев, А. А. Ионная избирательность клеточных мембран /
    А. А. Лев.~--- Ленинград : Наука, 1975.~--- 323 с.
\bibitem{bib:16} Антонов, В. Ф. Биофизика мембран / В. Ф. Антонов // Соросовский
    образовательный журнал.~--- 1996.~--- №~6.~--- С. 1--15.
\bibitem{bib:17} Боровягин, В. Л. Клеточные мембраны / В. Л. Боровягин //
    Биологические мембраны.~--- 1971.~--- №~4.~--- С. 746--766.
\bibitem{bib:18} Жулев, В. И. Исследование электрических процессов в клеточных
    структурах / В. И. Жулев, И. А Ушаков // Биомедицинская электроника.~---
    2001.~--- №~7.~--- С. 30--37.
\bibitem{bib:19} Кудряшов, Ю. Б. Биофизические основы действия микроволн /
    Ю. Б. Кудряшов, Э. Ш. Исмаилов, С. М. Зубкова.~--- Москва : МГУ, 1980.~---
    160 с.
\end{thebibliography}
