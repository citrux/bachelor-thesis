\pagestyle{empty}
\addcontentsline{toc}{chapter}{Список использованных источников}
\def\bibname{СПИСОК ИСПОЛЬЗОВАННЫХ ИСТОЧНИКОВ}
\begin{thebibliography}{99}
\bibitem{bib:0} Шеин, А. Г. Математические модели влияния СВЧ-излучения низкой
    интенсивности на пассивный транспорт ионов : монография / А. Г. Шеин, Н. В.
    Грецова ; ВолгГТУ.~--- Волгоград : ИУНЛ ВолгГТУ, 2012.~--- 148 c.
\bibitem{bib:13} Шеин, А. Г. Низкочастотные границы влияния СВЧ-излучения низкой
    интенсивности / А. Г. Шеин, Д. А. Барышев // Биомедицинская радиоэлектроника.~---
    2008.~--- №~4.~--- С. 4--8.
\bibitem{bib:14} Шеин, А. Г. Токи через мембрану с учетом наличия высокочастотных
    составляющих / А. Г. Шеин, Д. А. Барышев // Биомедицинская радиоэлектроника.~---
    2009.~--- №~4.~--- С. 4--9.
\bibitem{bib:1} Никулин, Р. Н. Физические механизмы воздействия СВЧ-излучения
    низкой интенсивности на биологические объекты : дис. \ldots\ канд. физ.-мат.
    наук: 01.04.04, 03.00.02 / Р. Н. Никулин.~--- Волгоград, 2004.~--- 129 с.
\bibitem{bib:3} Исмаилов, Э. Ш. Биофизическое действие СВЧ-излучений~---
    Москва : Энергоатомиздат, 1987~--- 144 с.
\bibitem{bib:23} Тамбиев, А. Х. Влияние КВЧ-излучения на транспортные свойства
    мембран у фотосинтезирующих организмов / А. Х. Тамбиев, Н. Н. Кирикова,
    Е. Н. Маркарова // Биомедицинская радиоэлектроника.~--- 1997.~--- №~4.~--- С.
    67--76.
\bibitem{bib:15} Чопчиян, А. С. О краевых задачах для уравнений Нернста--Планка
    и Пуассона / А. С. Чопчиян, Е. Н. Коржов // Системы управления и
    информационные технологии.~--- 2009.~--- №~38.~--- С. 200--203.
\bibitem{bib:24} Маркин, В. С. Индуцированный ионный транспорт / В. С. Маркин,
    Ю. А. Чизмаджев.~--- Москва : Наука, 1974.~--- 252 с.
\bibitem{bib:5} Рубин, А. Б. Биофизика. В 2 т. Т. 2. Биофизика клеточных
    процессов / А. Б. Рубин.~--- Москва : Высшая школа, 1999.~--- 464 с.
\bibitem{bib:25} Биофизика : учебник для студентов вузов / В. Ф. Антонов, А. М.
    Черныш, В. И. Пасечник, С. А. Вознесенский, Е. К. Козлова ; под ред. Антонова.~---
    3-е изд., испр. и доп.~--- Москва : Владос, 2006.~--- 287 с.
\bibitem{bib:26} Тиманюк, В. А. Биофизика : учебник для студентов вузов /
    В. А. Тиманюк, Е. Н. Животова.~--- Харьков : Золотые страницы, 2003.~--- 705 с.
\bibitem{bib:27} Плонси, Р. Биоэлектричество: Количественный подход / Р. Плонси,
    Р. Барр.~--- Москва : Мир, 1992.~--- 366 c.
\bibitem{bib:22} Биофизика / П. Г. Костюк, Д. М. Гродзинский,
    В. Л. Зима, И. С. Магура, Е. П. Сидорик, М. Ф. Шуба ; под ред. П. Г. Костюка.~--- Киев: Высшая
    школа, 1988.~--- 504 с.
\bibitem{bib:16} Иваницкий, Г. Р. Математическая биофизика клетки / Г. Р.
    Иваницкий, В. И. Кринский, Е. Е. Сельков.~--- Москва : Наука, 1987.~---
    308 c.
\bibitem{bib:4} Лев, А. А. Ионная избирательность клеточных мембран /
    А. А. Лев.~--- Ленинград : Наука, 1975.~--- 323 с.
\bibitem{bib:18} Антонов, В. Ф. Биофизика мембран / В. Ф. Антонов // Соросовский
    образовательный журнал.~--- 1996.~--- №~6.~--- С. 1--15.
\bibitem{bib:20} Боровягин, В. Л. Клеточные мембраны / В. Л. Боровягин //
    Биологические мембраны.~--- 1971.~--- №~4.~--- С. 746--766.
\bibitem{bib:19} Жулев, В. И. Исследование электрических процессов в клеточных
    структурах / В. И. Жулев, И. А Ушаков // Биомедицинская электроника.~---
    2001.~--- №~7.~--- С. 30--37.
\bibitem{bib:2} Кудряшов, Ю. Б. Биофизические основы действия микроволн /
    Ю. Б. Кудряшов, Э. Ш. Исмаилов, С. М. Зубкова~--- Москва : МГУ, 1980.~---
    160 с.
\end{thebibliography}
