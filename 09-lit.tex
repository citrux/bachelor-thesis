\addcontentsline{toc}{chapter}{Список использованных источников}
\def\bibname{СПИСОК ИСПОЛЬЗОВАННЫХ ИСТОЧНИКОВ}
\begin{thebibliography}{99}
\bibitem{bib:1} Никулин Р. Н. Физические механизмы воздействия СВЧ – излу-чения низкой интенсивности на биологические объекты [Электронный ре-сурс]: дис. … канд. физ.-мат. наук / Р. Н. Никулин; ВолгГТУ. − Волгоград, 2004. − 129 с. − Режим доступа: 1 электрон.опт. диск (CD-ROM).
\bibitem{bib:2} Кудряшов Ю. Б., Исмаилов Э. 111., Зубкова С. М. Биофизические основы действия микроволн. М.: Изд-во МГУ, 1980, 160 с.
\bibitem{bib:3} Исмаилов Э.Ш. Биофизическое действие СВЧ-излучений. М., Энергоатомиздат, 1987, 144с.
\bibitem{bib:4} Лев, А.А. Ионная избирательность клеточных мембран / А.А. Лев. – Л.: Наука, 1975. – 323 с.
\bibitem{bib:5} Рубин, А.Б. Биофизика: В 2-х кн.: Учеб.для биол. спец. вузов. Кн. 2. Биофизика клеточных процессов / А.Б. Рубин.– М.: Высшая школа, 1987. – 303 с.
\bibitem{bib:6} The influence of microwave radiation from cellular phone on fetal rat brain / J. Jing [and others] // Electromagnetic Biology and Medicine. 2012, Vol. 31, No. 1 , Pages 57-66
\bibitem{bib:7} J. R. Warchalewski1, J. Gralik. Influence of Microwave Heating on Biological Activities and Electrophoretic Pattern of Albumin Fraction of Wheat Grain // СerealСhemistry. 2010, Vol. 87, Nu. 1 P. 35 – 41.
\bibitem{bib:8} Bernard Е. P., Herman P. S. Further Observations on the Electrical Properties of Hemoglobin - Bound Water. - J. Phys Chem., 1969, v. 73, № 8, p. 2600-2610.
\bibitem{bib:9} Baranski S., Szmigielski S., Moneta J. Effects of mocrowave irradiation in vitro on cell membrane permeability. Biologic effects and Health Hazards of Microwave Radiation. Polish Med. Publ., Warshawa, 1974, p. 173-177
\bibitem{bib:10} Надарейшвили Г.Г. Комплексное воздействие ЭМП и ионизирующей радиации на трансмембранный перенос ионов в клетке // Изв. АН Грузии. 2006. № 3. С. 547-551.
\bibitem{bib:11} Девятков Н.Д. Влияние электромагнитного излучения ММ -диапазона длин волн на биологические объекты. - УФН, 1973, т. 10, вып. 3, с. 453 -454.
\bibitem{bib:12} Сорокина Т. П., Квашина О. П. Электронный учебник для
    ди-станционного обучения по курсу: физика и биофизика [Электронный
    ресурс]. Красноярск: ФГОУ ВПО Красноярский государственный аграрный
    университет, 2006.
    \url{http://www.kgau.ru/distance/etf_04/biophysics/index.htm}
\bibitem{bib:13} Шеин А.Г., Низкочастотные границы СВЧ излучения низкой интенсивности  / А.Г. Шеин, Д.А. Барышев // Биомедицинская радиоэлектроника. 2008. №4. С. 4 – 8
\bibitem{bib:14} Шеи н А.Г..  Токи через мембрану с учетом наличия высокоча-стотных составляющих  / А.Г. Шеин, Д.А. Барышев // 20-я Междунар. кон-фер.«СВЧ-техника и телекоммуникационные технологии», 13-17 сентября 2010 г. Севастополь, Крым, Украина, 2010. - C. 1155-1156
\bibitem{bib:15} Boronovsky S.E., Seraya I.P., NartsissovYa.R. A Brownian dynamic model of the glycine receptor chloride channel; effect of the position of charged aminoacidson ion membrane currents // IEE Proc.-Syst. Biol. 2006. V.153. №5. P.394-397.
\bibitem{bib:16} Бороновский С.Е., Нарциссов Я.Р. Электростатическая модель ионного канала глицинового рецептора // Научная сессия МИФИ-2006 Сборник научных трудов. 2006. Т.5. С.158-159.
\bibitem{bib:17} Boronovsky S.E., Nartsissov Ya.R. Probability simulator of enzyme activity and its application to description of transmembrane currents through glycine receptor // In book: Modern trends in Systems biology. Virtualmodelingandregulation. 2010. P.113-119.
\bibitem{bib:18} Антонов, В.Ф. Биофизика мембран [Текст]/В.Ф. Анто-нов//Соросовский образователь¬ный журнал.– 1996.– №6.– С. 1–15
\bibitem{bib:19} Исследование электрических процессов в клеточных структурах [Текст]/В.И. Жулев, И.А Ушаков//Биомедицинская электроника.– 2001.– №7.– С. 30–37
\bibitem{bib:20} Боровягин, В.Л. Клеточные мембраны [Текст]/ В.Л. Боровягин// Биологические мембраны.-1971.-№4.-С.746-766
\bibitem{bib:21} Chandler WK and Meves H. Slow changes in membrane permeability and long-lasting action potentials in axons perfused with fluoride solutions [Текст]/ J Physiol 211: 707–728, 1970.
\bibitem{bib:22} Костюк, П.Г. Биофизика [Текст]/П.Г. Костюк [и др.].– Киев: Высшая школа, 1988.– 504 с.
\bibitem{bib:23} Тамбиев А. Х. , Кирикова Н. Н. , Маркарова Е. Н. Влияние КВЧ-излучения на транспортные свойства мембран у фотосинтезирующих организмов. - Радиотехника, 1997, № 4
\bibitem{bib:24} Hodgkin, A., and Huxley, A. (1952): A quantitative description of membrane current and its application to conduction and excitation in nerve. J. Physiol. 117:500-544.
\bibitem{bib:25} Hille, B. (2001): Ionic Channels of Excitable Membranes. — (3rd ed.). — Sinauer Associates, Inc., Sunderland, MA.
\end{thebibliography}
