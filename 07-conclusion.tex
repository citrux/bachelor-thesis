\begin{center}
    ЗАКЛЮЧЕНИЕ
\end{center}
\addcontentsline{toc}{chapter}{Заключение}
В данной работе была предпринята попытка получить эквивалент уравнения
Нернста—Планка для нестационарного процесса. Было получено уравнение
\eqref{eq:nernst-plank-2_with_time}, которое может быть использовано для рассмотрения
нестационарного процесса переноса ионов в мембране, однако его недостатком является то,
что для его решения требуется явный вид профиля потенциала в мембране.
В качестве попытки обойти эту проблему, было получено уравнение
\eqref{eq:epic-equation}, которое включает в себя только одну неизвестную функцию.
Но и это уравнение в том виде, в котором оно получено, не может решить
поставленную задачу, так как для постановки краевой задачи требуется 3
граничных условия, в то время как у нас имеется только 2. Тем не менее,
если удастся решить эту проблему, то данное уравнение можно будет
использовать для описания транспорта ионов в мембране при наличии
высокочастотного поля.
