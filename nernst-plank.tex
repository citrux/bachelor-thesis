\documentclass{hedwork}
\usepackage[utf8]{inputenc}
\usepackage[russian]{babel}
\usepackage[derivative,vectors]{hedmaths}
\begin{document}
\tableofcontents
\section{Уравнение Нернста-Планка}
    Уравнение Нернста-Планка имеет вид
    \begin{equation}
        j = -\frac{z}{|z|}ukT\pder{n}{x} - |z|nue\pder{\phi}{x}.
        \label{eq:nernst-plank}
    \end{equation}
    Оно позволяет найти плотность ионного тока. При этом считается, что по всей
    мембране плотность тока одинакова, то есть рассматривается установившийся
    режим. Так как меня интересует переохдный процесс, приводящий к формированию
    равновесных распреелений концентрации и потенциала в мембране, то мне
    необходимо ввести в это уравнение время. Для этого можно воспользоваться
    уравнением непрерывности:
    \begin{equation}
        \pder{\rho}{t} + \divergence\vec{j} = 0.
    \end{equation}
    Учитывая, что \( \rho = nez \), для одномерного случая получим уравнение
    \begin{equation}
        \pder{n}{t} = \frac{ukT}{e|z|}\ppder{n}{x} +
            \frac{z}{|z|}u\pder{\phi}{x}\pder{n}{x} +
            \frac{z}{|z|}u\ppder{\phi}{x}n.
        \label{eq:nernst-plank_with_time}
    \end{equation}
    Попрбуем теперь узнать характерное время, необходимое на установление
    равновесных распределений концентраций. Для этого будем считать, что
    потенциал в начальный момент времени уже имеет равновесное
    распределение. Рассмотрим 2 простых случая, особенно часто употребляемых в
    биофизике:
    \begin{enumerate}
        \item \textbf{линейное распределение потенциала}\\
            в этом случае
            \[
                \pder{\phi}{x} = -E = \const
            \]
            и уравнение (\ref{eq:nernst-plank_with_time}) упрощается
            \begin{equation}
                \pder{n}{t} = \frac{ukT}{e|z|}\ppder{n}{x} -
                    \frac{z}{|z|}uE\pder{n}{x}.
            \end{equation}
        \item \textbf{линейное распределение концентрации}\\
            в этом случае сначала необходимо из уравнения
            (\ref{eq:nernst-plank}) получить равновесное распределение
            потенциала:
            \[
                n = n_{out} - \frac{n_{out} - n_{in}}{d} x,\ j = -\pder{z}{|z|}ukTn_1 -
                    |z|ue(n_{out}+n_1x)\pder{\phi}{x},
            \]
            \[
                \pder{\phi}{x} = -\frac{\phi_0}{\ln\frac{n_{out}}{n_{in}}}
                \frac{n_{in} - n_{out}}{d}
                \frac{1}{n_{out} - \frac{n_{out} - n_{in}}{d} x}.
            \]
            Подставляя в уравнение (\ref{eq:nernst-plank_with_time}), получаем
            \begin{gather}
                \pder{n}{t} = \frac{ukT}{e|z|}\ppder{n}{x} -
                \frac{z}{|z|}u\frac{\phi_0}{\ln\frac{n_{out}}{n_{in}}}
                \frac{n_{in} - n_{out}}{d}
                \frac{1}{n_{out} - \frac{n_{out} - n_{in}}{d} x}
                \pder{n}{x} + \nonumber \\
                + \frac{|z|}{z}u\frac{\phi_0}{\ln\frac{n_{out}}{n_{in}}}
                \frac{(n_{in} - n_{out})^2}{d^2}
                \frac{1}{(n_{out} - \frac{n_{out} - n_{in}}{d} x)^2}n.
            \end{gather}
    \end{enumerate}

\section{Приближение постоянного поля}
\subsection{Условие применимости}
    В приближении постоянного поля стационарное распределение концентраций,
    получаемая из уравнения Нернста-Планка, определяется зависимостью
    \[
        n(x) = \frac{(n_{out}e^\alpha - n_{in}) - (n_{out} -
        n_{in})e^{\alpha\frac{x}{d}}}{e^\alpha - 1},\ \alpha = \frac{zeEd}{kT}.
    \]
    Ионы в мембране создают поле \( E_i \), причем это поле действует и на
    сами ионы. Условием применимости может служить требование
    \[
        E_i^{max} \ll E.
    \]
    Определим поле \( E_i \). Для этого воспользуемся уравнением Максвелла
    \[
        \divergence\vec{E}_i = \pder{E_i}{x} = \frac{\rho}{\eps\eps_0}.
    \]
    Интегрируя его, получим
    \begin{gather}
        E_i(x) = E_i(0) + \frac{ez}{\eps\eps_0}\int_0^x n(\xi)d\xi = \nonumber\\
        = E_i(0) + \frac{ez}{\eps\eps_0}\left[
        \frac{n_{out}e^\alpha - n_{in}}{e^\alpha - 1}x - \frac{n_{out} -
        n_{in}}{e^\alpha - 1}\frac{d}{\alpha}{e^{\alpha\frac{x}{d}} - 1}
        \right].
    \end{gather}
    Так как поле создаётся плоскими слоями, в границах которых плотность заряда
    постоянна, то на краях мембраны поле внутренних ионов будет иметь разное
    направление, но одинаковую величину:
    \begin{equation}
        E_i(d) = -E_i(0) = E_i(0) + \frac{ez}{\eps\eps_0}\left[
        \frac{n_{out}e^\alpha - n_{in}}{e^\alpha - 1}d - \frac{n_{out} -
        n_{in}}{e^\alpha - 1}\frac{d}{\alpha}{e^\alpha - 1}
        \right].
    \end{equation}
    Отсюда находим \( E_i(0) \) и подставляем в выражение для поля:
    \begin{equation}
        E_i = \frac{ez}{\eps\eps_0}\left[
        \frac{n_{out}e^\alpha - n_{in}}{e^\alpha - 1}\left(x-\frac{d}{2}\right)
        - \frac{n_{out} - n_{in}}{e^\alpha - 1}\frac{d}{\alpha}
        \left(e^{\alpha\frac{x}{d}} - \frac{e^\alpha + 1}{2}\right)
        \right].
    \end{equation}
    Очевидно, что максимальное значение величина поля принимает вблизи краёв
    мембраны:
    \begin{equation}
        \boxed{
        E_i^{max} = \frac{ezd}{2\eps\eps_0\alpha(e^\alpha - 1)}\left\{
            n_{out}[(\alpha - 1)e^\alpha + 1] - n_{in}[\alpha + 1 - e^\alpha]
        \right\} \ll E.}
    \end{equation}
\subsection{Решение уравнения}
    Для удобства в уравнении
    \[
        \pder{n}{t} = \frac{ukT}{e|z|}\ppder{n}{x} -
            \frac{z}{|z|}uE\pder{n}{x}.
    \]
    введём следующие обозначения
    \[
        \frac{ukT}{e|z|} = D,\quad \frac{z}{|z|}uE = v.
    \]
    С учётом этого, уравнение принимает вид
    \[
        \pder{n}{t} = D\ppder{n}{x} - v\pder{n}{x}.
    \]
    Это уравнение конвективной диффузии. Поставим задачу Неймана для
    этого дифференциального уравнения в частных производных -- будем
    считать концентрации на краях мембраны постоянными, так как они
    определяются концентрациями в омывающих растворах, а начальное
    условие -- нулевым, то есть будем считать, что сначала мембрана
    свободна от ионов. Задача приобретает вид
    \begin{align*}
        & \pder{n}{t} = D\ppder{n}{x} - v\pder{n}{x},\ x\in(0,d)\\
        & n(0, t) = n_{out},\ t>0 \\
        & n(d, t) = n_{in},\ t>0 \\
        & n(x, 0) = 0,\ x\in(0,d).
    \end{align*}
    Для удобства решения обезразмерим её:
    \( x = \xi d, t = \tau d^2 / D \)
    \begin{align*}
        & \pder{n}{\tau} = \ppder{n}{\xi} -
            w\pder{n}{\xi},\ w = \frac{vd}{D},\ \xi\in(0,1) \\
        & n(0, \tau) = n_{out},\ \tau>0 \\
        & n(1, \tau) = n_{in},\ \tau>0 \\
        & n(\xi, 0) = 0,\ \xi\in(0,1).
    \end{align*}
    Теперь построим разностную схему для решения этого уравнения.
    Воспользуемся явной схемой
    \[
        \frac{n(\xi_i,\tau_{j+1}) - n(\xi_i, \tau_j)}{\Delta\tau} =
        \frac{n(\xi_{i+1},\tau_j) - 2n(\xi_i, \tau_j) +
        n(\xi_{i-1},\tau_j)}{\Delta\xi^2} -
        w\frac{n(\xi_{i+1},\tau_j) - n(\xi_{i-1}, \tau_j)}{2\Delta\xi}.
    \]
    Если теперь явно выразить значение на следующем временном шаге через
    значение на предыдущем, получим
    \begin{gather*}
        n(\xi_i,\tau_{j+1}) =
        [1-2r]n(\xi_i, \tau_j) +
        r\left(
            [1 - s]n(\xi_{i+1},\tau_j) + [1 + s]n(\xi_{i-1},\tau_j)
        \right),\\
        r = \frac{\Delta\tau}{\Delta\xi^2}, s = \frac{w\Delta\xi}{2}.
    \end{gather*}

\section{Приближение постоянного градиента концентрации}
\subsection{Условие применимости}
    Как и в приближении постоянного поля достаточно потребовать малости поля
    ионов в мембране по сравнению с внешним полем
    \[
        E = \frac{\phi_0}{\ln\frac{n_{out}}{n_{in}}}
            \frac{n_{in} - n_{out}}{d}
            \frac{1}{n_{out} - \frac{n_{out} - n_{in}}{d} x}.
    \]
    Внешнее поле принимает минимальное значение на том краю, где концентрация
    ионов больше:
    \[
        E_{min} = \frac{\phi_0}{\ln\frac{n_{out}}{n_{in}}}
            \frac{n_{in} - n_{out}}{d}\frac{1}{\max(n_{out}, n_{in})}.
    \]
    Для собственного поля ионов имеем
    \begin{gather}
        E_i(x) = E_i(0) + \frac{ez}{\eps\eps_0}\int_0^x n(\xi)d\xi = \nonumber\\
        = E_i(0) + \frac{ez}{\eps\eps_0}\left(
        n_{out} x - \frac{n_{out} - n_{in}}{d}\frac{x^2}{2}\right).
    \end{gather}
    Максимальное значение оно принимает на краях мембраны. Поле ионов имеет вид
    \begin{equation}
        E_i(x) = \frac{ez}{\eps\eps_0}\left[
        n_{out} \left(x-\frac{d}{2}\right) - \frac{n_{out} - n_{in}}{2d}
        \left(x^2 - \frac{d^2}{2}\right)
        \right].
    \end{equation}
    \begin{equation}
        \boxed{
        E_i^{max} = \frac{ezd}{\eps\eps_0}\frac{n_{in}+n_{out}}{4} \ll E_{min}.}
    \end{equation}
\end{document}
